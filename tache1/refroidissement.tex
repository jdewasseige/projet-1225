\section{Débit d'eau nécessaire}

Afin de maintenir les réacteurs à température constante, 
on refroidit ceux-ci avec de l'eau.
Il faut tout d'abord calculer la chaleur dégagée par jour par les réacteurs,
pour ensuite déterminer la quantité d'eau nécessaire pour absorber cette chaleur,
sachant que l'eau entre à $25 \si{\degreeCelsius}$ et 
est évacuée à $90 \si{\degreeCelsius}$.
Les calculs suivants sont présentés pour une quantité d'ammoniac de $1000\si{\tonne}$ par jour, 
la fonction \texttt{refroidissement} permet d'avoir le débit nécessaire (en litres par seconde) 
en fonction de la quantité de \ce{NH3} désirée.

\paragraph{In}

La chaleur dégagée par la réaction \ref{eq:ammoniac} correspond à la 
différence entre l'enthalpie de formation des produits et celle des réactifs. 

\begin{equation}
	\Delta H_{reaction} = \Delta H_{f, \ce{NH3}} 
	- \frac{1}{2} \Delta H_{f, \ce{N2}}
	- \frac{3}{2} \Delta H_{f, \ce{H2}}
	\label{eq:enthalpie_reaction}
\end{equation}

La réaction se passe à $500 \si{\degreeCelsius}$, 
on détermine donc les $\Delta H_{f}$ des différents 
composés à partir des valeurs des enthalpies standard
de formation\footnote{Atkins \& Jones - Chemical Principles}
(à $25 \si{\degreeCelsius}$)
et des capacités calorifiques molaires ($C_p$), gr\^ace 
à l'équation suivante.

\begin{equation*}
	\Delta H_{f, T2} = \Delta H_{f, T1} 
	+ \int_{T1}^{T2} C_p \, \dif{T}
\end{equation*}

Étant donné que nous travaillons avec des quantités importantes de matière,
nous ne pouvons pas considérer que $C_p$ est constant.
On sait que $C_p$ varie en fonction de la température 
selon une fonction $a + b T + c T^2$, 
où $a$, $b$ et $c$ sont des constantes\footnote{\url{http://www.edu.upmc.fr/
chimie/lc101-202-301/communs/public/capcalo.htm}}. 

Nous pouvons à présent calculer les enthalpies 
de formation à $500 \si{\degreeCelsius}$

\begin{equation}
	\Delta H_f = \Delta H_{f}^{\si{\degree}} + 
	\int_{298}^{773} (a + b T + c T^2) \, \dif{T}
	\label{eq:enthalpie_500C}
\end{equation}

À partir de l'équation \ref{eq:enthalpie_500C}, 
on obtient les valeurs suivantes

\begin{align*}
	\Delta H_{\ce{H2}} &= 14 \, \si{\kilo\joule\per\mole} \\
	\Delta H_{\ce{N2}} &= 14.19 \, \si{\kilo\joule\per\mole} \\
	\Delta H_{\ce{NH3}} &= -26.21 \, \si{\kilo\joule\per\mole} 
\end{align*}

Pour finir, on trouve l'enthalpie de réaction par l'équation \ref{eq:enthalpie_reaction}

\[
	\Delta H_{reaction} = -54.31 \, \si{\kilo\joule\per\mole}
\]

\paragraph{Out}

Le raisonnement pour la chaleur absorbée est similaire 
au précédent, sauf que cette fois ci, 
on considère que la capacité calorifique est constante.
Cette hypothèse peut être raisonnablement envisagée 
étant donné que\footnote{\url{http://kinetics.nist.gov/janaf/html/H-065.html}} 
$C_{p, 298 \si{\kelvin}} \approx 75.35 \, \si{\joule\per\kelvin\per\mole}$ 
et $C_{p, 368 \si{\kelvin}} \approx 75.78 \, \si{\joule\per\kelvin\per\mole}$.
La différence étant très petite, 
nous prendrons la moyenne de ces deux valeurs. 

La chaleur absorbée par une mole d'eau vaut exactement

\begin{equation}
	\Delta H = \int_{298}^{363} C_{p, \ce{H2O}} \, \dif{T}
	= 75.565 \cdot 65 = 4911.73 \, \si{\joule\per\mole}
\end{equation}

\paragraph{Débit}

Il s'agit maintenant de déterminer la quantité d'eau nécessaire par jour.
Il faut que

\[
	\frac{\Delta H_{in}}{\mathrm{jour}} 
	= - \frac{\Delta H_{out}}{\mathrm{jour}}
\]

En sachant qu'on produit $1000 \si{\tonne}$ de \ce{NH3} par jour
(cela équivaut à $5.88\e{7} \, \si{\mole}$),
et à partir des données trouvées ci-dessus, 
on trouve que

\begin{align*}
	n_{\ce{H2O}} &= \frac{n_{\ce{NH3}} \, \Delta H_{in}}{\Delta H_{out}} \\
	&= 6.5\e{8} \si{\mole} 
\end{align*}

Il faut donc $11703 \si{\tonne}$ d'eau par jour, 
ou encore 136 litres par seconde.
