\section{Bilan de masse} 

Nous faisons l'hypothèse que la réaction est parfaite, 
c'est-à-dire que tous les réactifs sont consommés 
et que l'on obtient uniquement le produit.

Il faut déterminer la quantité de dihydrogène (\ce{H2}) et d'azote (\ce{N2}) 
nécéssaire pour produire $1000 \si{\tonne}$ d'ammoniac (\ce{NH3}).

\begin{equation}
	\ce{\frac{3}{2} \, H2 + \frac{1}{2} \, N2 -> NH3} 
	\label{eq:ammoniac}
\end{equation}

La masse molaire de l'ammoniac vaut $17 \si{\gram\per\mole}$,
il faut donc en produire $5.88\e{7} \si{\mole}.$

On obtient les nombres de moles de réactifs à partir de la réaction pondérée. 

\begin{align*}
	m_{\ce{H2}}  = \frac{3}{2} \, n_{\ce{NH3}} \cdot M_{m, \ce{H2}} 
	= 1.765\e{8} \si{\gram} \\
	m_{\ce{N2}} = \frac{1}{2} \, n_{\ce{NH3}} \cdot M_{m, \ce{N2}} 
	= 8.235\e{8} \si{\gram}
\end{align*}

Les masses de \ce{H2} et de \ce{N2} valent donc 
respectivement $176.5 \si{\tonne}$ et $823.5 \si{\tonne}$.
Le bilan de masse est correct : si on fait la somme des masses des réactifs,
on retrouve la même masse produite.
