\documentclass[a4paper, oneside, 12pt]{article}

\usepackage{../custom}
\usepackage[francais]{babel}
\usepackage[utf8]{inputenc}


\title{Tâche 3}
\author{Groupe 1225}
\date{10 Décembre 2014}

\begin{document}

\maketitle

\section{Impact des produits utilisés ou rejetés lors du procédé}

\paragraph{Les gaz à effet de serre} 

Les gaz à effet de serre sont des composants gazeux qui absorbent le rayonnement infrarouge
émis par la surface terrestre et contribuant à l’effet de serre. 
L’impact principal de l’augmentation de leur concentration 
dans l’atmosphère est le réchauffement climatique. 
Celui-ci implique une forte hausse des températures terrestres, 
qui influence la fonte de la calotte glaciaire 
et modifie donc les conditions de l’écosystème.

\paragraph{Dioxyde de carbone (\ce{CO2})}

Le dioxyde de carbone est le principal gaz à effet de serre anthropique, 
c’est-à-dire produit par l’homme. 
Il représente $60\%$ de l’effet de serre anthropique et des pays industrialisés, 
pas moins de $80\%$ des émissions de gaz à effet de serre. 
La concentration actuelle de dioxyde de carbone 
est environ $30\%$ supérieure à celle avant l’industrialisation. 
Sa durée de séjour dans l’atmosphère est de plus ou moins un siècle. 
Dans notre schéma de production, nous émettons en moyenne $1950\si{\tonne}$ 
de dioxyde de carbone par jour, 
ce qui représente l’équivalent de $1/10$ du dioxyde de carbone rejeté 
en moyenne par jour par l’ensemble des voitures en Europe.

Nous remarquons qu’au plus nous réduisons la température du réformateur primaire, 
au moins les rejets de \ce{CO2} sont élevés. 
Ceci est dû au fait que le réformateur nécessite moins d’énergie
et donc la production de dioxyde de carbone du four est réduite. 
Cependant la plus grosse partie du rejet est due au mécanisme en lui même. 

De plus, une usine de production d’ammoniac requiert un nombre conséquent d’employés 
qui doivent se déplacer jusqu’à leur lieu de travail. 
Leur déplacement est aussi une source de dioxyde de carbone, 
mais la quantité rejetée est forte heureusement très faible 
par rapport aux émissions du processus de production.

\paragraph{Vapeur d'eau (\ce{H2O_{(g)}})}

La vapeur d’eau est le principal gaz à effet de serre 
puisqu’elle contribue pour $60\%$ à l’effet de serre planétaire, 
contribution qui monte jusqu’à $90\%$ si l’on tient compte des nuages. 
Heureusement, une très petite partie de cette vapeur d’eau est anthropique. 
La vapeur d’eau d’origine naturelle est responsable de l’effet de serre naturel, 
sans lequel la planète aurait une température moyenne de $-18\si{\degreeCelsius}$. 
De plus la vapeur d’eau ne reste en moyenne que quelques jours dans l’atmosphère, 
ce qui est très inférieur à la plupart des autres gaz à effet de serre. 
Les concentrations en vapeur d’eau n’ont pas changé par rapport à l’ère préindustrielle.
Nos rejets de vapeur d’eau ne sont donc pas problématiques,
ou enfin dans une très moindre mesure. 
On peut cependant remarquer sur nos graphes que la quantité d’eau nécessaire
au processus diminue très fortement lorsque la température du réformateur primaire augmente. 
Cela augmente donc le prix de production de l’ammoniac.
La firme devra donc faire des choix entre la diminution de ses coûts de production 
et l’impact écologique de ses émissions de dioxyde de carbone.

\paragraph{Méthane (\ce{CH4})}

Le méthane est un puissant gaz à effet de serre,
il a un impact $25$ fois plus puissant que le dioxyde de carbone
et son impact augmente avec le temps. 
Il reste en moyenne $12$ ans dans l’atmosphère.
Les émissions de méthane augmentent avec le temps, 
notamment à cause des réserves de méthanes dans les calottes glaciaires. 
Ces dernières se libèrent à cause du réchauffement climatique qui accélère la fonte des glaces. 

Nous ne rejetons pas directement du méthane, mais une fuite lors du transport 
ou du processus pourrait augmenter dramatiquement les émissions de gaz à effet de serre de l’usine.  
De même, lors des vidanges des réformateurs, 
une petite quantité de méthane pourrait s’en échapper.

\paragraph{Ammoniac (\ce{NH3})}

L’ammoniac est l’un des principaux responsables de l’acidification de l’eau 
et des sols ainsi qu’un facteur favorisant les pluies acides. 
Ce qui cause l’eutrophisation, c’est-à-dire le processus par lequel des nutriments 
s’accumulent dans un milieu terrestre ou aquatique. 
Ceci est dû à l’acidification du milieu, 
ce qui peut rendre certaines espèces plus vulnérables à la pollution et aux maladies. 
	
L’ammoniac est toxique pour la plupart des plantes 
et des animaux dès que sa concentration dépasse la vitesse de désintoxication des espèces. 
Il peut par exemple résulter pour les plantes, d’un ralentissement de croissance, 
d’une moindre résistance aux changements de températures ou aux parasites. 
Pour les animaux, les risques se trouvent plutôt du côté des maladies. 

Comme pour le méthane, nous ne rejetons pas directement de l’ammoniac, mais une fuite,
les vidanges ou autre peuvent vite entrainer un rejet important.
En plus de son impact environnemental, ces rejets sont néfastes pour l’homme.

\paragraph{Argon (\ce{Ar})}

L’argon rejeté par le processus ne provoque aucun dommage environnemental connu. 
Ce n’est pas un gaz à effet de serre, 
il n’influence pas la faune et flore et n’est pas énuméré comme polluant marin 
par le \textsc{dot} (Department of Transportation). 
Il comporte cependant un risque pour la santé, car il s’évapore très rapidement 
et entraine la saturation de l’air avec un risque sérieux dans les espaces confinés.
 de l’ammoniac.
La firme devra donc faire des choix entre la diminution de ses coûts de production 
et l’impact écologique de ses émissions de dioxyde de carbone.

\end{document}

