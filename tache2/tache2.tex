\documentclass[a4paper, oneside, 12pt]{article}

\usepackage{../custom}

\title{Tâche 2}
\author{Groupe 1225}
\date{\today}

\begin{document}

\maketitle

\section{Démarrage}

Lors de la première tâche, nous analysions l'installation dans son ensemble, mais de manière excessivement simplifiée. Dans cette section, nous nous concentrerons sur l'explicitation de la dernière étape pour ensuite la simuler sur le logiciel $ASPEN PLUS$. Nous ne désirons plus la considérer comme une boîte noire, et l'avons donc découpée en un certain nombre de processus intermédiaires.
Le gaz composé d'$N_2$, d'$H_2$ et d'$Ar$ passe par un réchauffeur et par un compresseur pour atteindre respectivement la température et la pression voulue. Il va ensuite dans le réacteur où se déroule la réaction. Le produit est refroidit dans une installation prévue à cette fin, pour arriver enfin dans un séparateur où on extrait l'ammoniac.
Il semblait intéressant de créer un circuit de recyclage pour réutiliser le mélange de $N_2$ et de $H_2$ issu du processus (la réaction n'est pas complète et il reste donc des réactifs). Malheureusement, il y existe également de l'argon qu'on ne peut séparer du reste. En effet, sa température d'ébullition est bien trop proche de celle des autres composés et les moyens à mettre en œuvre en seraient très couteux et/ou à faible rendement. 
Notre groupe a donc décidé de faire une purge dans le circuit de recyclage afin d'éviter que l'argon ne s'accumule. Cette purge ne doit pas être trop grande sinon la quantité perdue des réactifs serait conséquente. La condition à respecter est donc d'avoir un flux entrant d'Argon égal à celui de sortie:


\end{document}

