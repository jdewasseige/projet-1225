\documentclass[a4paper, oneside, 12pt]{article}

\usepackage{../custom}

\usepackage[backend=biber]{biblatex}
\addbibresource{biblio.bib}

\title{Tâche 8}
\author{Groupe 1225}
\date{3 Décembre 2014}

\begin{document}

\maketitle

\section*{Enjeux environnementaux liés à la production d'ammoniac}

La production d'hydrogène dans le reformage primaire lors de la réaction

\begin{equation*}
	\ce{CH4_{(g)} + H2O_{(g)} <=> CO_{(g)} + 3 \, H2_{(g)}} 
\end{equation*}

fait intervenir du méthange gazeux sur deux plans : il en faut en produits de réaction, 
mais également dans le four pour que celui-ci puisse approvisionner la réaction en énergie, étant donné que celle-ci est endothermique. 

Le méthane n'est a priori pas un composé très polluant à créer. C'est un hydrocarbure simple à produire à l'aide de processus naturels, notamment la biométhanisation. 
Cependant, c'est un gaz à effet de serre, avec des propriétés environ 9 fois plus fortes que le \ce{CO2}. 
Si il arrive qu'il y ait une fuite dans la premier réacteur ou dans le four, la libération de méthane pourrait engendrer une forte pollution de l'air.\cite{electrolyse}

\printbibliography
\end{document}

