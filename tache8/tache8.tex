\documentclass[a4paper, oneside, 12pt]{article}

\usepackage{../custom}

%\usepackage[backend=biber]{biblatex}
%\addbibresource{biblio.bib}

\title{Tâche 8}
\author{Groupe 1225}
\date{\today}

\begin{document}

\maketitle

\section*{Enjeux environnementaux liés à la production d'ammoniac}

La production de \ce{H2} dans le reformage primaire lors de la réaction
\[ \ce{CH4_{(g)} + H2O_{(g)} <=> CO_{(g)} + 3 \, H2_{(g)}} \]
fait intervenir du méthange gazeux sur deux plans : il en faut comme produits de réaction,
mais également dans le four pour que celui-ci puisse approvisionner la réaction en énergie,
étant donné que celle-ci est endothermique. 

Le méthane n'est pratiquement pas polluant au niveau de sa fabrication. 
En effet, c'est un hydrocarbure relativement simple à produire à l'aide 
de processus naturels, notamment la biométhanisation. 
Cependant, c'est un gaz à effet de serre encore plus fort que le \ce{CO2}.
S'il arrive qu'il y ait une fuite dans la premier réacteur ou dans le four, 
la libération de méthane pourrait engendrer une forte pollution de l'air.

Au niveau du four, suite à la combustion du méthane : 
\[ \ce{CH4_{(g)} + 2 \, O2_{(g)} -> CO2_{(g)} + 2 \, H2O_{(g)}} \] 
nous constatons un rejet de \ce{H2O} et de \ce{CO2}, 
tous deux étant des gaz à effet de serre. 
La vapeur d'eau ne restant pas longtemps dans l'atmosphère,
sa contribution au réchauffement climatique est donc négligeable. 
Par contre, le dioxyde de carbone pose plus de problèmes. 

Lors de la synthèse de \ce{H2} dans les reformages primaire et secondaire, 
et dans le réacteur Water-Gas-Shift : 

\[ \ce{CH4_{(g)} + H2O_{(g)} <=> CO_{(g)} + 3 \, H2_{(g)}} \]

\[ \ce{CO_{(g)} + H2O_{(g)} <=> CO2_{(g)} + H2_{(g)}} \]

\[ \ce{2 \, CH4_{(g)} + O2_{(g)} -> {2} \, CO_{(g)} + 4 \, H2_{(g)}} \]
	
\[ \ce{CO_{(g)} + H2O_{(g)} -> CO2_{(g)} + H2_{(g)}} \]

nous rejetons du \ce{CO2} supplémentaire en grandes quantités. 
Typiquement, pour produire 1500t de \ce{NH3} avec le procédé Haber, 
nous rejetons $1947,145\si{\tonne}$ de \ce{CO2}.

Nous remarquons que l'intégralité du rejet de \ce{CO2} provient directement 
ou indirectement de la production de \ce{H2}, le \ce{N2} étant extrait 
directement depuis l'air ambiant.

Le procédé de Haber est assez néfaste pour l'environnement. 
La production de \ce{H2} entraîne le dégagement de quantités 
importantes de \ce{CO2} dans l'atmosphère.
Nous allons analyser plusieurs pistes possibles pour tenter 
d'améliorer l'impact environnemental.

\section*{Alternatives au vaporeformage}

Plusieurs solutions (réactions toutes endothermiques) 
sont possibles afin de remplacer le vaporeformage.

La gazéification à l'eau permet de synthétiser de l'hydrogène 
gazeux à partir de matières carbonnées et de vapeur d'eau.
L'apport énergétique doit être sous forme de chaleur.

\[ \ce{C_{(g)} + H2O_{(g)} -> CO_{(g)} + H2_{(g)}} \]

\[ \ce{CO_{(g)} + H2O_{(g)} <=> CO2_{(g)} + H2_{(g)}} \]

L'électrolyse de l'eau permet notamment de produire de l'hydrogène 
en transformant de l'énergie électrique en énergie chimique.
L'apport énergétique doit être sous forme électrique.

\[ \ce{H2O_{(g)} <=> \frac{1}{2} \, O2_{(g)} + H2_{(g)}} \]

Nous allons ici nous intéresser plus en détails au cas de l'électrolyse de l'eau. 

D'après nos résultats en laboratoire sur la réaction d'électrolyse, 
il ne faudrait pas moins de $5,7 \si{\giga\watt}$ 
pour produire $266,32\si{\tonne}$ de \ce{H2} par jour, 
c'est-à-dire la quantité d'hydrogène nécessaire 
à la fabrication de $1500\si{\tonne}$ de \ce{NH3} par jour. 
L'électrolyse est un procédé qui consomme énormément d'énergie électrique. 
Comme nous allons le voir ci-dessous, il n'est intéressant que si 
l'électricité provient de sources d'énergie renouvelables dégageant 
très peu de dioxyde carbone, telle que l'énergie éolienne ou encore hydraulique.

En considérant les rejets standards de \ce{CO2} par \si{\kilo\watt\hour} électrique 
en Belgique, qui sont de $290\si{\gram/\kilo\watt\hour}$,
nous calculons une émission de dioxyde de carbone de l'ordre de $39 672\si{\tonne}$
par jour pour produire $1500\si{\tonne}$ d'ammoniac. 
Nous obtenons donc une émission de \ce{CO2} deux fois 
supérieure à celle calculée avec le vaporeformage.

Ce résultat est évident puisque nous avons pris en compte 
les émissions standards de \ce{CO2} par kWh.
Par contre, si nous décidons de produire notre électricité à l'aide de sources écologiques,
nous allons considérablement diminuer notre émission de \ce{CO2}, 
en passant de $290\si{\gram/\kilo\watt\hour}$ à seulement $5\si{\gram/\kilo\watt\hour}$, 
valeur standard pour les énergies éolienne et hydraulique. 
Dans ce cas-là, le rejet de \ce{CO2} pour produit les $5,7\si{\giga\watt}$ nécessaires 
ne sera que de $684\si{\tonne}$ par jour, 
contre $1947,15\si{\tonne}$ pour le vaporeformage. 
\newline

On remarque finalement que le procédé d'électrolyse de l'eau 
est nettement préférable d'un point de vue environnemental au vaporeformage, 
mais entraîne à côté de cela un coût plus important pour la production d'hydrogène,
en raison des besoins considérables en énergie électrique.

\end{document}

