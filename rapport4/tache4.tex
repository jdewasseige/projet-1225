\documentclass[a4paper,oneside,12pt]{article}

\NeedsTeXFormat{LaTeX2e}
\ProvidesPackage{custom}[2014/05/11 Custom Package]

\usepackage[utf8]{inputenc}
\usepackage[T1]{fontenc}
\usepackage[francais]{babel}

\usepackage[version=3]{mhchem}
\usepackage{chemfig}

\usepackage{amsmath}
\usepackage{amsthm}
\usepackage{amsfonts}

\usepackage{graphicx} 
\usepackage[top=3cm, bottom=3cm, left=3cm , right=3cm]{geometry}
%\usepackage{setspace} %doublespace, onehalfspace
\usepackage{siunitx}

\usepackage{tikz}
\usetikzlibrary{positioning}
\usetikzlibrary{shapes,arrows}

\usepackage{tabularx}
\usepackage{url} 
\usepackage{tocloft} %spacing in list of figures
\usepackage{listings} %input code
%\usepackage{multibbl} %multiplebibliography
\usepackage{hyperref}
\usepackage[babel=true]{csquotes}
\usepackage{listings}
\usepackage{color}

\usepackage{epstopdf}

\usepackage{caption}
\usepackage{subcaption}
\usepackage{float}

\newcommand{\dif}[1]{\mathrm{d}#1}
\newcommand{\e}[1]{\cdot 10^{#1}}

\endinput


\title{Tache 4 : Mini-HAZOP du noeud autour du réacteur de synthèse d'ammoniac}
\author{Groupe 1225}
\date{20 Novembre 2014}

\begin{document}

\maketitle

\section*{Question 1}

6 gaz sont présents dans le réacteur. Certains ne présentent que peu de danger, d'autres peuvent provoquer d'importants dégats en cas de combustion.

- L'hydrogène est un gaz extrêmement inflammable et peut provoquer de grosses explosions en cas de combustion. C'est le principal danger lié à sa présence.
En cas d'accumulation d'hydrogène dans une pièce ou un batiment, sa présence peut provoquer un environnement déficient en oxygène et donc provoquer la suffocation.

- L'argon est présent naturellement dans l'air et n'est dangereux qu'en quantités importantes. Il peut alors provoquer la suffocation.

- L'azote est également naturellement présent dans l'air et n'est pas plus dangereux que l'argon.

- L'ammoniac est inflammable et peut donc provoquer une explosion en cas d'accumulation et de combustion.

- L'helium peut provoquer l'asphyxie en cas de concentration trop élevée.

- Le méthane est inflammable et présente un danger d'explosion en cas d'accumulation et de combustion. Il y a également un risque d'asphyxie en cas d'accumulation simple.

\section*{Question 2}

\section*{Question 3}

\section*{Question 4}

\section*{Question 5}

\end{document}
