\documentclass[a4paper,oneside,12pt]{article}

\NeedsTeXFormat{LaTeX2e}
\ProvidesPackage{custom}[2014/05/11 Custom Package]

\usepackage[utf8]{inputenc}
\usepackage[T1]{fontenc}
\usepackage[francais]{babel}

\usepackage[version=3]{mhchem}
\usepackage{chemfig}

\usepackage{amsmath}
\usepackage{amsthm}
\usepackage{amsfonts}

\usepackage{graphicx} 
\usepackage[top=3cm, bottom=3cm, left=3cm , right=3cm]{geometry}
%\usepackage{setspace} %doublespace, onehalfspace
\usepackage{siunitx}

\usepackage{tikz}
\usetikzlibrary{positioning}
\usetikzlibrary{shapes,arrows}

\usepackage{tabularx}
\usepackage{url} 
\usepackage{tocloft} %spacing in list of figures
\usepackage{listings} %input code
%\usepackage{multibbl} %multiplebibliography
\usepackage{hyperref}
\usepackage[babel=true]{csquotes}
\usepackage{listings}
\usepackage{color}

\usepackage{epstopdf}

\usepackage{caption}
\usepackage{subcaption}
\usepackage{float}

\newcommand{\dif}[1]{\mathrm{d}#1}
\newcommand{\e}[1]{\cdot 10^{#1}}

\endinput


\title{Tache 4 : Mini-\textsc{hazop} du noeud autour du réacteur de synthèse d'ammoniac}
\author{Groupe 1225}
\date{20 Novembre 2014}

\begin{document}

\maketitle

\section*{Question 1}

6 gaz sont présents dans le réacteur. Certains ne présentent que peu de danger, d'autres peuvent provoquer d'importants dégats en cas de combustion.
\begin{itemize}
\item L'hydrogène est un gaz extrêmement inflammable et peut provoquer de grosses explosions en cas de combustion. C'est le principal danger lié à sa présence.
En cas d'accumulation d'hydrogène dans une pièce ou un batiment, sa présence peut provoquer un environnement déficient en oxygène et donc provoquer la suffocation.

\item  L'argon est présent naturellement dans l'air et n'est dangereux qu'en quantités importantes. Il peut alors provoquer la suffocation.

\item  L'azote est également naturellement présent dans l'air et n'est pas plus dangereux que l'argon.

\item  L'ammoniac est inflammable et peut donc provoquer une explosion en cas d'accumulation et de combustion.

\item  L'helium peut provoquer l'asphyxie en cas de concentration trop élevée.

\item  Le méthane est inflammable et présente un danger d'explosion en cas d'accumulation et de combustion. Il y a également un risque d'asphyxie en cas d'accumulation simple.
\end{itemize}

\section*{Question 2}

\section*{Question 3}

\section*{Question 4}
% Pourquoi n’y a-t-il pas de soupape de sécurité ou de disque de rupture (les deux types 
% de dispositifs servent à protéger un équipement ou une ligne contre les surpressions) 
% sur le réacteur de synthèse du NH3 ?

Un disque de rupture est un dispositif de sécurité qui sert à protéger les installations 
contre les surpressions.
La réaction de synthèse de l'ammoniac est la suivante

\[
	\ce{N2_{(g)} + 3H2_{(g)} <=> 2NH3_{(g)}}
\]

On remarque immédiatement qu'il y a une diminution du nombre de moles de gaz (d'un facteur 2) 
lorsque de l'ammoniac est produit. La loi des gazs parfait nous indique que la pression
exercée par un gaz est directement proportionnelle à son nombre de moles.
Une surpression n'est pas envisageable pour cette réaction, 
il n'y a donc pas besoin de disque de rupture sur ce réacteur.

\section*{Question 5}
% Pourquoi y a-t-il des disques de rupture sur l’échangeur 124-MC ?

L'échangeur 124-C est un dispositif permettant de transférer de la chaleur d'un fluide
vers un autre, sans que ceux-ci ne se mélangent.
On a donc un flux \emph{chaud} et un flux \emph{froid}. Le flux froid va recevoir de 
l'énergie thermique et sa température va augmenter, il faut alors prévoir des disques 
de rupture au cas où la température dépasse une certaine limite qui produirait une 
supression du gaz et pourrait endommager, voire déchirer, la paroi.
Théoriquement, le flux chaud n'a pas besoin de disque de rupture parce que le gaz qu'il 
transporte ne peut que diminuer de volume. Cependant, on peut toujours considérer le cas
d'un apport soudain et imprévu de chaleur comme une explosion extérieure, qui produirait
une augmentation de température et par conséquent de pression. 
Il devient alors nécessaire d'avoir des disques de rupture dans les deux sens de l'échangeur.

\end{document}
