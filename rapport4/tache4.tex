\documentclass[a4paper,oneside,12pt]{article}

\NeedsTeXFormat{LaTeX2e}
\ProvidesPackage{custom}[2014/05/11 Custom Package]

\usepackage[utf8]{inputenc}
\usepackage[T1]{fontenc}
\usepackage[francais]{babel}

\usepackage[version=3]{mhchem}
\usepackage{chemfig}

\usepackage{amsmath}
\usepackage{amsthm}
\usepackage{amsfonts}

\usepackage{graphicx} 
\usepackage[top=3cm, bottom=3cm, left=3cm , right=3cm]{geometry}
%\usepackage{setspace} %doublespace, onehalfspace
\usepackage{siunitx}

\usepackage{tikz}
\usetikzlibrary{positioning}
\usetikzlibrary{shapes,arrows}

\usepackage{tabularx}
\usepackage{url} 
\usepackage{tocloft} %spacing in list of figures
\usepackage{listings} %input code
%\usepackage{multibbl} %multiplebibliography
\usepackage{hyperref}
\usepackage[babel=true]{csquotes}
\usepackage{listings}
\usepackage{color}

\usepackage{epstopdf}

\usepackage{caption}
\usepackage{subcaption}
\usepackage{float}

\newcommand{\dif}[1]{\mathrm{d}#1}
\newcommand{\e}[1]{\cdot 10^{#1}}

\endinput


\author{Groupe 1225}
\date{20 Novembre 2014}
\title{Tache 4 : Mini-HAZOP du noeud autour du réacteur de synthèse d'ammoniac}

\begin{document}

\maketitle

\section*{Question 1}

Hydrogène : Extrêmement inflammable, danger d'explosion. En présence élevée dans l'air, peut provoquer une environnement déficient en oxygène et donc provoquer la suffocation. 

Des métaux communs sont des catalyseurs pour la réaction de combustion.

Argon : Agent asphyxiant simple : Peut entrainer des maux de têtes, voir la suffocation et la mort en concentrations élevées. Si présent sous forme liquide, danger de gelure en cas de contacte avec la peau ou les yeux.

CO2 : Si grandes quantités relachées en milieu confiné ou mal aéré, risque d'asphyxie.

CO : Intoxication en milieu fermé (remplace l'oxygène dans les globules rouges). Peut causer la mort en réduisant l'approvisionnement en oxygène du cerveau.

Methane : Risques d'asphyxie, danger d'explosion extrême.

Ammoniac : Risque de combustion. Danger d'explosion si présence de Chlorine.

Oxygène : Favorise la combustion d'autres composés.

Eau : Danger de brulure car eau à haute température. Explosion de vapeur d'eau si de l'eau froide entre en contact avec un liquide chaud dont la température est fort suppérieur à la température d'ébullition de l'eau et l'eau froide rentrent en contact.

\section*{Question 2}

\section*{Question 3}

\section*{Question 4}

\section*{Question 5}

\end{document}