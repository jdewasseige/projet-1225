\documentclass[a4paper, oneside, 12pt]{article}

\usepackage{custom}

\title{Rapport 1}
\author{Groupe 1225}
\date{\today}

\begin{document}

\maketitle

\section{Bilan de matière}

Nous faisons l'hypothèse que la réaction est parfaite, 
c'est-à-dire que tous les réactifs sont consommés 
et que l'on obtient uniquement le produit.

Il faut déterminer la quantité de dihydrogène (\ce{H2}) et d'azote (\ce{N2}) 
nécéssaire pour produire $1000 \si{\tonne}$ d'ammoniac (\ce{NH3}).

\begin{equation}
	\ce{\frac{3}{2} \, H2 + \frac{1}{2} \, N2 -> NH3} 
	\label{eq:ammoniac}
\end{equation}

La masse molaire de l'ammoniac vaut $17 \si{\gram\per\mole}$,
il faut donc en produire $5.88 \cdot 10^7 \si{\mole}.$

On obtient les nombres de moles de réactifs à partir de la réaction pondérée. 

\begin{align*}
	m_{\ce{H2}}  = \frac{3}{2} \, n_{\ce{NH3}} * M_{m, \ce{H2}} 
	= 1.765 \cdot 10^8  \si{\gram} \\
	m_{\ce{N2}} = \frac{1}{2} \, n_{\ce{NH3}} * M_{m, \ce{N2}} 
	= 8.235 \cdot 10^8 \si{\gram}
\end{align*}

Les masses de \ce{H2} et de \ce{N2} valent donc 
respectivement $1.765 \si{\tonne}$ et $8.235 \si{\tonne}$.
Le bilan de masse est correct : si on fait la somme des masses des réactifs,
on retrouve la même masse produite.


\section{Débit d'eau nécessaire}

Afin de maintenir les réacteurs à température constante, 
on refroidit ceux-ci avec de l'eau.
Il faut tout d'abord calculer la chaleur dégagée par jour par les réacteurs,
pour ensuite déterminer la quantité d'eau nécessaire pour absorber cette chaleur,
sachant que l'eau entre à $25 \si{\degreeCelsius}$ et 
est évacuée à $90 \si{\degreeCelsius}$. 

\paragraph{In}

La chaleur dégagée par la réaction \ref{eq:ammoniac} correspond à la 
différence entre l'enthalpie de formation des produits et celle des réactifs. 

\begin{equation}
	\Delta H_{reaction} = \Delta H_{f, \ce{NH3}} 
	- \frac{1}{2} \Delta H_{f, \ce{N2}}
	- \frac{3}{2} \Delta H_{f, \ce{H2}}
	\label{eq:enthalpie_reaction}
\end{equation}

La réaction se passe à $500 \si{\degreeCelsius}$, 
on détermine donc les $\Delta H_{f}$ des différents 
composés à partir des valeurs des enthalpies standard
de formation\footnote{Atkins \& Jones - Chemical Principles}
(à $25 \si{\degreeCelsius}$)
et des capacités calorifiques molaires ($C_p$), gr\^ace 
à l'équation suivante.

\begin{equation}
	\Delta H_{f, T2} = \Delta H_{f, T1} 
	+ \int_{T1}^{T2} C_p \, \mathrm{d}T
	\label{eq:enthalpie_temp}
\end{equation}

Étant donné que nous travaillons avec des quantités importantes de matière,
nous ne pouvons pas considérer que $C_p$ est constant.
On sait que $C_p$ varie en fonction de la température 
selon une fonction $a + b T + c T^2$, 
où $a$, $b$ et $c$ sont des constantes
\footnote{\url{http://www.edu.upmc.fr/
chimie/lc101-202-301/communs/public/capcalo.htm}}. 

Nous pouvons à présent calculer les enthalpies 
de formation à $500 \si{\degreeCelsius}$

\begin{equation*}
	\Delta H_f = \Delta H_{f}^{\si{\degree}} + 
	\int_{298}^{773} (a + b T + c T^2) \, \mathrm{d}T
\end{equation*}

\paragraph{Out}

Le raisonnement pour la chaleur absorbée est similaire 
au précédent.
La différence d'enthalpie vaut exactement

\[
	\int_{298}^{363} C_{p, \ce{H2O}} \, \mathrm{d}T
\]

Il faut que

\[
	\frac{\Delta H_{out}}{\mathrm{jour}} 
	= - \frac{\Delta H_{in}}{\mathrm{jour}}
\]


\end{document}

