\documentclass[a4paper, oneside, 12pt]{article}

\usepackage{custom}

\title{Rapport 1}
\author{Groupe 1225}
\date{\today}

\begin{document}

\maketitle

\section{Bilan de matière}

Nous faisons l'hypothèse que la réaction est parfaite, 
c'est-à-dire que tous les réactifs sont consommés 
et que l'on obtient uniquement le produit.

Il faut déterminer la quantité de dihydrogène (\ce{H2}) et d'azote (\ce{N2}) 
nécéssaire pour produire $1000 \si{\tonne}$ d'ammoniac (\ce{NH3}).

\begin{equation}
	\ce{\frac{3}{2} \, H2 + \frac{1}{2} \, N2 -> NH3} 
	\label{eq:ammoniac}
\end{equation}

La masse molaire de l'ammoniac vaut $17 \si{\gram\per\mole}$,
il faut donc produire $5.88 \cdot 10^7 \si{\mole}.$

On obtient les nombres de moles de réactifs à partir de la réaction pondérée. 

\begin{align*}
	m_{\ce{H2}}  =\frac{3}{2} \, n_{\ce{NH3}} * M_{m, \ce{H2}} = 1.765 \cdot 10^8  \si{\gram} \\
	m_{\ce{N2}} = \frac{1}{2} \, n_{\ce{NH3}} * M_{m, \ce{N2}} = 8.235 \cdot 10^8 \si{\gram}
\end{align*}

Les masses de \ce{H2} et de \ce{N2} valent donc 
respectivement $1.765 \si{\tonne}$ et $8.235 \si{\tonne}$.
Le bilan de masse est correct : si on fait la somme des masses des réactifs,
on retrouve la même masse produite.

\end{document}

