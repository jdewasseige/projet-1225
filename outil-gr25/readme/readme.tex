\documentclass[a4paper,oneside,11pt]{article}

\usepackage{../../custom}

\title{\textsc{readme} de l'outil de gestion}
\author{Groupe 1225}
\date{\today}

\newcommand{\fun}[1]{\texttt{#1}}

\begin{document}

\maketitle

\section{Utilisation}

Lancer la fonction \fun{startOutil}.

\paragraph{Remarque} Il se peut que pour certaines valeurs de $m_\ce{NH3}$,
une erreur de type ``\textit{Could not extract individual solutions. 
Returning a MuPAD set object.}'' survienne lors de l'appel à \fun{main}. 
Pour palier à ce problème il suffit de changer la valeur 
de $m_\ce{NH3}$ d'une unité.

\section{Fonctionnement}

Voici une liste de l'ensemble des fonctions utilisées dans notre 
outil de gestion.

\begin{itemize}
	\item \fun{airePSV}
	\item \fun{analyseParametrique}
	\item \fun{environnement}
	\item \fun{getCoefficients} 
	\item \fun{getDeltaH\_and\_S}
	\item \fun{getEqConstantsRef} 
	\item \fun{getHovenMasses}
	\item \fun{getMassesDetails}
	\item \fun{getMolarMasses}
	\item \fun{getTubesNumber}
	\item \fun{main}
	\item \fun{printHovenDetails}
	\item \fun{printMassesDetails}
	\item \fun{purge}
	\item \fun{purgeP}
	\item \fun{purgeT}
	\item \fun{refroidissement}
	\item \fun{simulation}
	\item \fun{solveG}
	\item \fun{startOutil}
\end{itemize}


\paragraph{Limite de validité}
Les coefficients utilisés dans les équations de Shomate ne sont 
cependant valables que pour une certaine gamme de températures 
(on retrouve les températures acceptables pour chaque composé 
en commentaire dans le fichier \fun{getCoefficients}). 
On peut donc s'attendre à une certaine erreur sur les résultats 
si l'utilisateur entre une température non réaliste. 

Afin de palier à ce soucis, nous avons créé une fonction \fun{myAssert}. 
Celle-ci est une variante de la fonction \fun{Assert} 
et permet d'afficher un message d'avertissement, 
tout en continuant l'éxécution du programme.


\end{document}
