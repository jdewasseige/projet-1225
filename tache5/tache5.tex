% Quelle est la pression normale de stockage ?
\paragraph{Q1} La pression normale de stockage est de $7.8 \barg$ à $20\si{\celsius}$ 
(température normale de stockage).

% Quelle sera la pression de stockage en été (30°C) ?
\paragraph{Q2} La pression de stockage en été est de $10.625 \barg$ à $30\si{\celsius}$.

% Quel sera la pression maximale de tarage de la soupape de sécurité ?
\paragraph{Q3} La pression maximale de tarage de la soupape de sécurité vaut 
\[ 121\% \cdot p_{\text{design}} = 18.15\barg \]

% Dimensionner la soupape pour cette pression de tarage.
%     -	Quelle sera la pression durant la décharge ?
%     -	Quelle sera la température du liquide durant la décharge via la soupape ?
%     -	Quelle sera la taille de la soupape nécessaire ?
\paragraph{Q4} 
La pression durant la décharge sera de $19.16325 \si{\bar}$. 
Cette pression est fort élevée car nous considérons le cas d'un incendie au niveau du réservoir. 
La température du gaz durant la décharge sera de $50\si{\celsius}$. 
Nous aurons besoin d'une soupape d'une section de $730\si{\milli\meter\squared}$ 
ou $1.13 \, \text{sqinch}$. Ceci correspond à une PSV de type $2J3$
\[ C = 0.03948 * \sqrt(1.33 * (2 / 2.33)^(2.33 / 0.33)) \]
\[ W = 43200 * 1 * 143.634^(0.82) \]
\[ A = (W / (C * 0.975 * 20.071 * 1 * 1)) * \sqrt(323.15/17) \]

% Si la pression de design de l’équipement était de 20 barg, 
% quel serait l’effet d’augmenter la pression de tarage de 5 bar et de la porter à 20 barg  ?
\paragraph{Q5} 
Si la pression de design est à $20\barg$, nous pouvons augmenter la pression de tarage à \[ 121\% \cdot p_{\text{design}} = 24.2\barg \]
Nous recalculons A avec la pression et température plus élevée, et obtenons que la soupape doit avoir une section de $590\si{\milli\meter\squared}$
ce qui nous donne $0.9118 \, \text{sqinch}$. Ceci correspond à une PSV de type $2J3$ comme celle trouvée précédemment. Il n'y a donc pas d'intérêt
à augmenter la pression de design dans ces conditions.

% Pour la première pression de tarage, quelle est l’influence d’isoler thermiquement 
% le tank avec un isolant tel que le coefficient d’échange avec l’extérieur 
% soit réduit à une valeur de 10 W/m2.K ? 
\paragraph{Q6}
Nous isolons la cuve avec un isolant réduisant le coefficient d'échange de la cuve avec l'extérieur à 
$10 \si{\watt}/\si{\meter\squared}*\si{\kelvin}$
On réduit donc le facteur environnemental à $F = 0.15$. 
Notre $W$ va donc être fortement réduit, et nous pouvons donc réduire la taille de la soupape à 
$110\si{\milli\meter\squared}$ ou $0.17 \, \text{sqinch}$. Cette PSV est du type $1E2$. L'ajout de l'isolant réduit donc fortement la taille de
la soupape requise.

