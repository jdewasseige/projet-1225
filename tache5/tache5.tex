Pour la t\^ache 5, il nous était demandé de dimensionner une soupape de sécurité
pour un tank contenant de l'ammoniac. Pour celà, nous avons recu quelques informations:

\begin{itemize}

\item Forme du tank: cylindrique vertical à extrémités hémisphériques. Tank au sol.
\item Hauteur totale du tank: $12 \si{\metre}$
\item Niveau de \ce{NH3} liquide dans le tank: $8\si{\metre}$
\item Diamètre du tank: $6 \si{\metre}$
\item Température normale de stockage: $20 \si{\degreeCelsius}$
\item Pression de design: $15 \barg$
\item Cp/Cv du \ce{NH3}: $1.33$; Facteur de compressibilité Z: $1.0$
\item La soupape sera une soupape conventionnelle et la contrepression sera nulle.
\item L’usine est munie de systèmes de drainages des fuites 
	et d’un équipement moderne de lutte contre l’incendie.
\end{itemize}

Nous avons également à notre disposition deux graphes donnant respectivement 
les pressions de stockage de l'ammoniac et de l'enthalpie de vaporisation 
en fonction de la température. 

% Quelle est la pression normale de stockage ?
\section{Pression normale de stockage} 
La pression normale de stockage est de $7.8 \barg$ à $20\si{\celsius}$ 
(température normale de stockage). Nous avons obtenus ces résultats sur base des
graphiques que nous avions recu.

% Quelle sera la pression de stockage en été (30°C) ?
\section{Pression de stockage en été} 
La pression de stockage en été est de $10.625 \barg$ à $30\si{\celsius}$.

% Quel sera la pression maximale de tarage de la soupape de sécurité ?
\section{Pression maximale de tarage de la soupape} 
La pression maximale de tarage de la soupape de sécurité vaut 
\[ 121\% \cdot p_{\text{design}} = 18.15 \barg \]

% Dimensionner la soupape pour cette pression de tarage.
%     -	Quelle sera la pression durant la décharge ?
%     -	Quelle sera la température du liquide durant la décharge via la soupape ?
%     -	Quelle sera la taille de la soupape nécessaire ?
\section{Dimensionnement de la soupape} 
La pression durant la décharge sera de $19.16325 \si{\bar}$. 
Cette pression est fort élevée car nous considérons le cas d'un incendie au niveau du réservoir. 
La température du gaz durant la décharge sera de $50\si{\celsius}$. 
Nous aurons besoin d'une soupape d'une section de $730\si{\milli\meter\squared}$ 
ou $1.13 \, \text{sqinch}$. Ceci correspond à une PSV de type $2J3$
\[ C = 0.03948 \cdot \sqrt{1.33 \cdot \frac{2}{2.33}^{\frac{2.33}{0.33}}} \]
\[ W = 43200 \cdot 1 \cdot 143.634^{0.82} \]
\[ A = (W / (C \cdot 0.975 \cdot 20.071 \cdot 1 \cdot 1)) \cdot \sqrt{\frac{323.15}{17}} \]

% Si la pression de design de l’équipement était de 20 barg, 
% quel serait l’effet d’augmenter la pression de tarage de 5 bar et de la porter à 20 barg  ?
\section{Effet de l'augmentation de la pression de tarage de 5 à 20 barg pour une pression de design de 20 barg} 
Si la pression de design est à $20\barg$, nous pouvons augmenter la pression de tarage à \[ 121\% \cdot p_{\text{design}} = 24.2\barg \]
Nous recalculons A avec la pression et température plus élevée, et obtenons que la soupape doit avoir une section de $590\si{\milli\meter\squared}$
ce qui nous donne $0.9118 \, \text{sqinch}$. Ceci correspond à une PSV de type $2J3$ comme celle trouvée précédemment. Il n'y a donc pas d'intérêt
à augmenter la pression de design dans ces conditions.

% Pour la première pression de tarage, quelle est l’influence d’isoler thermiquement 
% le tank avec un isolant tel que le coefficient d’échange avec l’extérieur 
% soit réduit à une valeur de 10 W/m2.K ? 
\section{Isolation thermique du tank}
Nous isolons la cuve avec un isolant réduisant le coefficient d'échange de la cuve avec l'extérieur à 
$10 \si{\watt}/\si{\meter\squared} \cdot \si{\kelvin}$
On réduit donc le facteur environnemental à $F = 0.15$. 
Notre $W$ va donc être fortement réduit, et nous pouvons donc réduire la taille de la soupape à 
$110\si{\milli\meter\squared}$ ou $0.17 \, \text{sqinch}$. Cette PSV est du type $1E2$. L'ajout de l'isolant réduit donc fortement la taille de
la soupape requise.

