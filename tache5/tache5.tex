\documentclass[a4paper,oneside,12pt]{article}

\usepackage{../custom}

\title{T\^ache 5}
\author{Groupe 1225}
\date{\today}

\newcommand{\barg}{\si{\bar}\text{g}} 

\begin{document}

\maketitle

% Quelle est la pression normale de stockage ?
\paragraph{Q1} La pression normale de stockage est de $7.5 \barg$ à $20\si{\celsius}$ 
(température normale de stockage).

% Quelle sera la pression de stockage en été (30°C) ?
\paragraph{Q2} La pression de stockage en été est de $10 \barg$ à $30\si{\celsius}$.

% Quel sera la pression maximale de tarage de la soupape de sécurité ?
\paragraph{Q3} La pression maximale de tarage de la soupape de sécurité vaut 
\[ 105\% \cdot p_{\text{design}} = 16.8\si{\bar} = 15.79\barg \]

% Dimensionner la soupape pour cette pression de tarage.
%     -	Quelle sera la pression durant la décharge ?
%     -	Quelle sera la température du liquide durant la décharge via la soupape ?
%     -	Quelle sera la taille de la soupape nécessaire ?

% Si la pression de design de l’équipement était de 20 barg, 
% quel serait l’effet d’augmenter la pression de tarage de 5 bar et de la porter à 20 barg  ?

% Pour la première pression de tarage, quelle est l’influence d’isoler thermiquement 
% le tank avec un isolant tel que le coefficient d’échange avec l’extérieur 
% soit réduit à une valeur de 10 W/m2.K ? 


\end{document}

