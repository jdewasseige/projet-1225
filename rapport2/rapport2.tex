\documentclass[a4paper, oneside, 12pt]{article}

\usepackage{../custom}

\title{Rapport 2}
\author{Groupe 1225}
\date{\today}

\begin{document}

\maketitle

\section{Calculs énergétiques}

Pour réaliser nos calculs, il est nécessaire de prendre 
en compte la variation possible de 
deux facteurs: $m$ la capacité du plant(en tonnes par jour) 
et $T$ la température à la sortie du
réacteur de reformage à la vapeur de méthane.
Nous obtiendrons donc des formules où ces variables sont des paramètres.

La première réaction analysée est la suivante:

\begin{equation*}
	\ce{CH4 + H2O -> CO + 3H2}
\end{equation*}

Afin de connaître l'énergie demandée par cette réaction,
il nous faut trouver son avancement. Celle-ci n'est en effet pas complète.
Pour ce faire, nous devons commencer par obtenir la constante $K$ 
de la réaction. Or, elle peut être donnée par la relation:

\begin{equation}
	\ln{(K(T))} = \ln{(K(298.15))} - 
	\frac{\Delta H^{\circ}}{R}(\frac{1}{T} - \frac{1}{298.15})
\end{equation}

L'enthalpie de la réaction à une température est égale 
à la différence pondérée des enthalpies de formation:

\begin{equation}
	\Delta_{reac} H^{\circ} (T)= \Delta_{form,\ce{CO}} H^{\circ}(T)
	+\Delta_{form, \ce{H2}} H^{\circ} (T) - \Delta_{form, \ce{CH4}} H^{\circ} (T) 
	-\Delta_{form, \ce{H2O}} H^{\circ} (T)
\end{equation}

La différence d'enthalpie de formation d'une molécule entre deux températures est
égale à l'intégrale de sa capacité calorifique (à pression fixe) entre ces deux températures.
On utilise donc cette formule en partant de la température standard:$\Delta_{form}H°(T)= \Delta_{form}H°(298.15)+\int_298.15^T Cp(T) \, \mathrm dT$ et pouvons remettre notre équation sous la forme suivante:

\begin{equation}
	\Delta_{réac}H°(T)=\Delta_{réac}H°(298.15)+\sum_i \int_{298.15}^T \nu_i Cp_i(T) \, \mathrm dT
\end{equation}

Nous utilisons alors l'approximation suivante pour les différents $C_P$\footnote{http://www.edu.upmc.fr/
chimie/lc101-202-301/communs/public/capcalo.htm}, 
chacun étant sous la forme $a+bT+cT^2$:

\begin{tabular}{|l|c|c|r|}
  \hline
  Gaz & $a$ & $b\e{3}$ & $c\e{6}$ \\
  \hline
  \ce{CH4} & 14,23 & 75,3 & -18,00\\
  \ce{H2O} & 30,13 & 10,46 & 0 \\
  \ce{CO} & 27,62 & 5,02 & 0\\
  \ce{H2} & 29,30 & -0,84 & 2,09\\
  \hline
\end{tabular}

On a alors:
\begin{equation}
	\sum_i \int_298.15^T \nu_i Cp_i(T) \, \mathrm dT=\int_{298.15}^T 71.16-8.326 10^{-2}T+2.427 10^{-5}T^2 \, \mathrm dT
\end{equation}
Une fois l'expression intégrée, on peut simplifier dans l'équation pour trouver l'enthalpie de réaction en fonction de la température:

\begin{equation}
	\Delta_{réac}H°(T)=188441.8653+71.16T-4.163 10^-2 T^2 + 8.09 10^-6 T^3
\end{equation}

Maintenant que l'on dispose de cela, il ne nous faut plus qu'une valeur de K à une certaine température pour l'obtenir comme une fonction de T. Pour trouver cette valeur particulière, on utilise la formule $K(T)=exp(-\frac{\DeltaG°(T)}{RT})$ à la température standard. On trouve facilement $\DeltaG°$ grâce aux valeurs des énergies libres de formation (obtenues dans un livre de référence\footnote{Chimie Physique de Atkins, DE BOECK} dont on fait la différence pondérée. Le résultat est: $K(298.15)=1.2597*10^-25$.\\
Nous pouvons enfin injecter ces résultats dans notre équation de départ:
\begin{equation}
	ln(K(T))=ln(K(298.15))-\frac{\DeltaH°(T)}{R}(\frac{1}{T}-\frac{1}{298.15})
\end{equation}

Et en simplifiant (attention, le grand nombre de simplification influe sur la précision):

\begin{equation}
	K(T)=exp(\frac{-22664.245}{T}+10.128+0.0337T+1.5797 10^{-5}T^2+3.263 10^{-9}T^3)
\end{equation}

\section{Fonctionnement de l'outil de calcul}

Notre outil de calcul a été réalisé via matlab et prends en considération deux variables : 
la température du réacteur de reformage primaire et la quantité d'ammoniac 
que l'on désire produire en 24H. 

A partir de la quantité d'ammoniac que nous désirons 
produire, l'outil calcule les quantités d'azote, d'hydrogène et d'oxygène 
nécessaires aux différentes réactions en considérant que
les réactions respectent les hypothèses données (toutes les réactions sauf 
le reformage primaire sont complètes et les différents composants sont présents en 
quantités stoechiométriques). L'oxygène et l'azote étant fournis par l'air,
l'outil nous donne directement la quantité d'air qui sera
nécessaire. Nous connaissons également la quantité d'hydrogène produite dans 
le reformage secondaire de par les hypothèses. 
Il nous reste donc à déterminer la quantité de méthane nécessaire 
pour produire le reste de l'hydrogène requis. 
\\
La réaction se déroulant dans le reformage primaire est incomplète et 
globalement endothermique (deux réactions se déroulent au même moment).
Le K de la réaction va donc varier suivant la température du réacteur.
Nous calculons donc d'abord le K de la réaction afin de 
connaitre la quantité de méthane nécessaire et la quantité d'hydrogène produite,
puis nous calculons l'énergie utilisée par la réaction.
Une fois l'énergie nécessaire connue, nous pouvons déterminer la quantité 
de méthane qui doit être fournie au four pour que le réacteur reste à température constante.

\end{document}
