\documentclass[a4paper, oneside, 12pt]{article}

\usepackage{../custom}
\usepackage{float}

\title{Rapport 2}
\author{Groupe 1225}
\date{\today}

\begin{document}

\maketitle

\section{Introduction}

Ce second rapport est une extension et une amélioration du rapport rendu en S2 et a pour but d'étudier plus en détail la production d'ammoniac. Il comprend un flow sheet qui permet de voir les flux de matières en les différentes étapes de la synthèse par le procédé Haber-Bosh. Il comprend aussi un calcul détaillé des aspects énergétiques dans le réformateur primaire. S’y trouve ensuite une explication du fonctionnement de l’outil de calcul réalisé sur MATLAB (ainsi que l'outis de calcul en annexe). Celui-ci permet le calcul des flux entrants et sortants en fonction de la quantité d'ammoniac voulue dans la journée ainsi que de la température dans le réformateur primaire. Il calcul également les aspects énergétiques lors des différentes étapes. 
Et enfin ce rapport se termine par une analyse de l'effet du changement des variables que sont la température et la quantité d'ammoniac produite sur le bilan de matière.

\section{Flow-sheet}

\begin{tikzpicture}
[node distance = 7em]

\tikzstyle{decision} = [diamond, draw, fill=blue!20, 
    text width=4.5em, text badly centered, node distance=3cm, inner sep=0pt]
\tikzstyle{block} = [rectangle, draw, fill=blue!20, 
    text width=15em, text centered, rounded corners, minimum height=4em]
\tikzstyle{line} = [draw, -latex']
\tikzstyle{cloud} = [draw, ellipse,fill=red!20, node distance=3cm,
    minimum height=2em]
    
\node [block] (primaire) {Reformage primaire (T)\\
\ce{CH4 + H2O <=> CO + 3H2} \\
\ce{CO + H2O <=> CO2 + H2}};
\node [cloud, left = 2em of primaire] (H2O) {\ce{H2O}};
\node [block, right = 2em of primaire] (four) {Four: Combustion \ce{CH4} (rendement 75\%) (\simeq 1300K)\\
\ce{CH4 + 2O2 -> CO2 + 2H2O}};
	\node [cloud, below = 2em of four] (O2) {\ce{O2}};
	\node [cloud, above = 2em of four] (CH4) {\ce{CH4}};
\node [block, below of=primaire] (secondaire) {Reformage secondaire (\simeq 1200K)\\
\ce{2CH4 + O2 -> 2CO + 4H2}};
	\node [cloud, right = 1em of secondaire] (air) {Air (21\% \ce{O2} 78\% \ce{N2} 1\% \ce{Ar})};
	\node [block, below of=secondaire] (wgs) {Water-Gas-Shift (\simeq 500-700K)\\
\ce{CO + H2O -> CO2 + H2}};
\node [block, below of=wgs] (abso) {Absorbtion du \ce{CO2} et compression};
	\node [decision, right = 2em of abso] (CO22) {\ce{CO2}};
	\node [decision, left = 2em of abso] (H2O2) {\ce{H2O}};
	\node [block, below of=abso] (synthese) {Synthèse \ce{NH3} et séparation de l'argon \\
(Sortie réacteur: 270bar, 750K) \\
\ce{3H2 + N2 <=> 2NH3}};
\node [decision, below of=synthese] (NH3) {\ce{NH3}};
\node [decision, right = 2em of synthese] (argon) {\ce{Ar}};


	\path [line] (primaire) -- node[anchor = east] {\ce{H2} \ce{CH4} \ce{CO2} \ce{CO} \ce{H2O}}(secondaire);
\path [line] (four) -- node[anchor = south] {ENERGIE}(primaire);
    \path [line,dashed] (CH4) -- (four);
    \path [line,dashed] (CH4) -| (primaire);
    \path [line,dashed] (O2) -- (four);
    \path [line] (secondaire) -- node[anchor = east] {\ce{H2} \ce{N2} \ce{CO2} \ce{CO} \ce{Ar} \ce{H2O}}(wgs);
    \path [line] (wgs) -- node[anchor = east] {\ce{H2} \ce{H2O} \ce{CO2} \ce{N2} \ce{Ar}}(abso);
    \path [line] (abso) -- node[anchor = east] {\ce{H2} \ce{N2} \ce{Ar}}(synthese);
\path [line,dashed] (synthese) -- (NH3);
\path [line,dashed] (synthese) -- (argon);
\path [line,dashed] (air) -- (secondaire);
\path [line,dashed] (H2O) -- (primaire);
\path [line,dashed] (abso) -- (CO22);
\path [line,dashed] (abso) -- (H2O2);

\end{tikzpicture}

\section{Calculs énergétiques}

Pour réaliser nos calculs, il est nécessaire de prendre 
en compte la variation possible de 
deux facteurs: $m$ la capacité du plant(en tonnes par jour) 
et $T$ la température à la sortie du
réacteur de reformage à la vapeur de méthane.
Nous obtiendrons donc des formules où ces variables sont des paramètres.

La première réaction analysée est la suivante:

\begin{equation*}
	\ce{CH4 + H2O <-> CO + 3H2}
\end{equation*}

Afin de connaître l'énergie demandée par cette réaction,
il nous faut trouver son avancement. Celle-ci n'est en effet pas complète.
Pour ce faire, nous devons commencer par obtenir la constante $K$ 
de la réaction. Or, elle peut être donnée par la relation:

\begin{equation}
	\ln{(K(T))} = \ln{(K(298.15))} - 
	\frac{\Delta H^{\circ}}{R}(\frac{1}{T} - \frac{1}{298.15})
\end{equation}

L'enthalpie de la réaction à une température est égale 
à la différence pondérée des enthalpies de formation:

\begin{equation}
	\Delta_{reac} H^{\circ} (T)= \Delta_{form,\ce{CO}} H^{\circ}(T)
	+\Delta_{form, \ce{H2}} H^{\circ} (T) - \Delta_{form, \ce{CH4}} H^{\circ} (T) 
	-\Delta_{form, \ce{H2O}} H^{\circ} (T)
\end{equation}

La différence d'enthalpie de formation d'une molécule entre deux 
températures est égale à l'intégrale de sa capacité calorifique 
(à pression fixe) entre ces deux températures.
On utilise donc cette formule en partant de la  température 
standard: $\Delta_{form}H°(T)= \Delta_{form}H°(298.15)+\int_298.15^T Cp(T) \, \mathrm dT$ et pouvons 
remettre notre équation sous la forme suivante:

\begin{equation}
	\Delta_{réac}H°(T)=\Delta_{réac}H°(298.15)+\sum_i \int_{298.15}^T \nu_i Cp_i(T) \, \mathrm dT
\end{equation}

Nous utilisons alors l'approximation suivante pour les différents $C_P$\footnote{http://www.edu.upmc.fr/
chimie/lc101-202-301/communs/public/capcalo.htm}, 
chacun étant sous la forme $a+bT+cT^2$:

\begin{tabular}{|l|c|c|r|}
  \hline
  Gaz & $a$ & $b\e{3}$ & $c\e{6}$ \\
  \hline
  \ce{CH4} & 14,23 & 75,3 & -18,00\\
  \ce{H2O} & 30,13 & 10,46 & 0 \\
  \ce{CO} & 27,62 & 5,02 & 0\\
  \ce{H2} & 29,30 & -0,84 & 2,09\\
  \hline
\end{tabular}

On a alors:
\begin{equation}
	\sum_i \int_298.15^T \nu_i Cp_i(T) \, \mathrm dT=\int_{298.15}^T 71.16-8.326 10^{-2}T+2.427 10^{-5}T^2 \, \mathrm dT
\end{equation}
Une fois l'expression intégrée, on peut simplifier dans l'équation pour trouver l'enthalpie de réaction en 
fonction de la température:

\begin{equation}
	\Delta_{réac}H°(T)=188441.8653+71.16T-4.163 10^-2 T^2 + 8.09 10^-6 T^3
\end{equation}

Maintenant que l'on dispose de cela, il ne nous faut plus qu'une valeur de K à une certaine température 
pour l'obtenir comme une fonction de T. Pour trouver cette valeur particulière, on utilise la 
formule $K(T)=exp(-\frac{\DeltaG°(T)}{RT})$ à la température standard. On trouve facilement $\DeltaG°$ grâce 
aux valeurs des énergies libres de formation (obtenues dans un livre de référence\footnote{Chimie Physique de 
Atkins, DE BOECK} dont on fait la différence pondérée. Le résultat est: $K(298.15)=1.2597*10^-25$.\\
Nous pouvons enfin injecter ces résultats dans notre équation de départ:
\begin{equation}
	ln(K(T))=ln(K(298.15))-\frac{\DeltaH°(T)}{R}(\frac{1}{T}-\frac{1}{298.15})
\end{equation}

Et en simplifiant (attention, le grand nombre de simplification influe sur la précision!):

\begin{equation}
	K(T)=exp(\frac{-22664.245}{T}+10.128+0.0337T+1.5797 10^{-5}T^2+3.263 10^{-9}T^3)
\end{equation}

Faisant exactement de la même manière, on peut obtenir la constante K de la deuxième équation. Tous 
les détails calculatoires sont en annexe sous la forme de code matlab.

\section{Bilan de matière}

Une partie de la t\^ache 1 consiste à réaliser un outil de gestion
pour pouvoir déterminer les quantités de matières première nécessaires,
lorsque 2 paramètres varient. 
Les paramètres sont la température à la sortie de réacteur primaire ainsi
que la quantité d'ammoniac produite par jour. 
On les notera respectivement $T$ et $m_{\ce{NH3}}$.

À travers cette section, nous décrirons différentes réactions et nous 
supposons que le lecteur sait quelle réaction a lieu à quel moment ainsi 
que l'ordre des réactions durant tout le processus.
Un flow-sheet simplifié de l'ensemble du processus détaillant chacune
des réactions se trouve à la section \ref{sec:flow}.

Nous connaissons la masse d'ammoniac à produire par jour, 
et par conséquent son nombre de moles ($n_{\ce{NH3}} = m_{\ce{NH3}}/M_{\ce{NH3}}$).
Note que les nombres de moles seront généralement écrits
en fonction de $m_{\ce{NH3}}$.

À partir de la réaction suivante,

\begin{equation}
	\ce{\frac{3}{2} \, H2 + \frac{1}{2} \, N2 -> NH3} 
	\label{eq:ammoniac}
\end{equation}

on détermine les quantités de \ce{H2} et de \ce{N2} nécessaires.

\begin{align}
	n_{\ce{H2}} = \frac{3}{34} \, m_{\ce{NH3}} \\
	n_{\ce{N2}} = \frac{1}{34} \, m_{\ce{NH3}}
\end{align}

Or la seule source d'azote est lorsque l'air entre dans le réacteur secondaire.
En connaissant les proportions de l'azote, de l'oxygène et de l'argon dans 
l'air, on peut facilement obtenir les quantités de \ce{O2} et de \ce{Ar}.

\begin{align}
	n_{air} = \frac{1}{0.78} \, n_{\ce{N2}} = \frac{25}{663} \, m_{\ce{NH3}} \\
	n_{\ce{O2}} = 0.21 \, n_{air} = \frac{7}{884} \, m_{\ce{NH3}} \\
	n_{\ce{Ar}} = 0.01 \, n_{air} = \frac{1}{2652} \, m_{\ce{NH3}}
\end{align}

Il nous reste maintenant à déterminer les quantités de \ce{CH4} et de \ce{H20}.
On voit que la seule réaction nécessitant de l'oxygène est celle du 
réformage secondaire. Ceci nous permet de conna\^itre toutes les quantités 
associées à cette réaction.

\begin{equation}
	\ce{2CH4 + O2 -> 2CO + 4H2}
	\label{eq:ref_sec}
\end{equation}

Dès lors, 

\begin{align}
	n_{\ce{CO}-3} = n_{\ce{CH4}-3} 
	= 2 \, n_{\ce{O2}} = \frac{7}{442} \, m_{\ce{NH3}} \\
	n_{\ce{H2}-3} = 4 \, n_{\ce{O2}} = \frac{7}{221} \, m_{\ce{NH3}}
\end{align}

On peut ensuite utiliser le fait que les deux premières réactions soient 
à l'équilibre. C'est ici qu'intervient le paramètre $T$. On va donc commencer
par calculer les $\Delta G^{\circ}_{reaction}$ pour déterminer les $K(T)$.

On détermine ceux-ci via la relation 
\begin{equation}
	\Delta G_{reaction} = \Delta H_{reaction} - T \, \Delta S_{reaction}
\end{equation}

On rappelera que 
\begin{align}
	\Delta H_{f, T2} = \Delta H_{f, T1} + \int_{T1}^{T2} C_p \, \dif{T} \\
	\Delta S_{f, T2} = \Delta S_{f, T1} + \int_{T1}^{T2} 
	\frac{C_p}{T} \, \dif{T}
\end{align}

Les valeurs des coefficients\footnote{\url{http://www.edu.upmc.fr/
chimie/lc101-202-301/communs/public/capcalo.htm}} $a + b T + c T^{2}$, 
ainsi que les enthalpies et entropies standard\footnote{\url{http://kinetics.
nist.gov/janaf/html/H-065.html}} 
sont trouvées dans les liens ci-dessous.

La relation suivante

\begin{equation}
	K(T) = \exp{(\frac{- \Delta G}{R \, T})}
\end{equation}

nous permet de trouver K_{c}.

Et enfin dans la réaction du Water-Gas-Shift:

\begin{equation*}
	\ce{CO + H2O -> CO2 + H2}
\end{equation*}

\begin{equation}
	n_{CO}/jour=n_{{CH_4}_{secondaire}}+n_{{CH_4}_{primaire}}
\end{equation}

\begin{equation}
	n_{CO_2}/jour=n_{CO}/jour
\end{equation}

A présent, intéressons nous à l'avancement des réactions dans le réacteur primaire. Pour la première réaction, 
à l'équilibre, on a une quantité suivante de réactifs pour un certain \xi_{1}:
	\begin{equation}
	\ce{CH4 + H2O <-> CO + 3H2}
	\end{equation}
	\\
\begin{tabular}{|l|c|c|c|r|}
  \hline
  \ce{CH4} & \ce{H2O} & \ce{CO} & \ce{H2} & n_gaz \\
  \hline
  n_{CH_4} & n_{H_2O} & 0 & 0 & n_{CH_4} + n{H_2O}\\
  n_{CH_4}-\xi_{1} & n_{H_2O}-\xi_{1} & \xi_{1} & $3 \xi_{1}$ & n_{CH_4} + n_{H_2O} + $2 \xi_{1}$\\
  \hline
\end{tabular}
\\
Pour la deuxième équation, avec un certain \xi_{2}:\\
	\begin{equation}
	\ce{CO + H2O <-> CO2 + H2}
	\end{equation}
	\\
\begin{tabular}{|l|c|c|c|r|}
  \hline
  \ce{CO} & \ce{H2O} & \ce{CO2} & \ce{H2} & n_gaz \\
  \hline
   \xi_{1} & n_\ce{H_2O}-\xi_{1} & 0 & $3\xi_{1}$ & n_{H_2O}+$3\xi_{1}$\\
   \xi_{1}-\xi_{2} & n_\ce{H_2O}-\xi_{1}-\xi_{2} & \xi_{2} & $3\xi_{1} +\xi_{2}$ & n_{H_2O}+$3\xi_{1}$\\
  \hline
\end{tabular}
\\
Grâce à cela, on peut obtenir une expression pour les constantes d'équilibres K_1 et K_2:

$$K_1=\frac{{p_{tot}}^2}{{p_0}^2} \frac{1}{(n_\ce{CH4}+n_\ce{H2O}+2\xi_{1})^2} \frac{(\xi_{1}-\xi_{2})(3\xi_{1}+\xi_{2})^{-3}}{(n_\ce{CH4}-\xi_{1})(n_\ce{H2O}-\xi_{1}-\xi_{2})}$$
$$K_2=\frac{\xi_{2} (3\xi_{1}+\xi_{2}}{(\xi_{1}-\xi_{2})(n_\ce{H2O}-\xi_{1}-\xi_{2})}$$

\begin{equation}
K_1=\frac{{p_{tot}}^2}{{p_0}^2} \frac{1}{(n_\ce{CH4}+n_\ce{H2O}+2\xi)^2} \frac{(\xi-\eta)(3\xi+\eta)^{-3}}{(n_\ce{CH4}-\xi)(n_\ce{H2O}-\xi-\eta)}
\end{equation}
\begin{equation}
K_2=\frac{\eta (3\xi+\eta}{(\xi-\eta)(n_\ce{H2O}-\xi-\eta)}
\end{equation}

Ce que l'on peut égaler aux valeurs des constantes trouvées dans la section précédente.

De plus, nous pouvons égaler ce qui sort du réacteur primaire à ce qui rentre dans le réacteur secondaire 
(trouvé précédemment):
\begin{equation}
n_\ce{CH4}-\xi_1=\frac{7m_\ce{NH3}}{442}
\end{equation}

\begin{equation}
3\xi_1+\xi_2=\frac{9m_\ce{NH3}}{221}
\end{equation}

On obtient donc finalement un système de quatre équations à 4 inconnues - où 2 d'entre elles ne sont autres que les débits 
d'\ce{H2O} et de \ce{CH4} -, que nous pouvons résoudre sur Matlab pour plus de facilité (voir annexe). 
Comme prévu, ces quantités dépendent de T et de la masse d'ammoniac voulue à la sortie.

\section{Fonctionnement de l'outil de calcul}

L'outil de calcul a été réalisé en matlab et prends en considération deux variables : la température du 
réacteur primaire (en K) et la quantité d'ammoniac (en tonnes) que l'on souhaite produire en 24h. 
\\

Nous commençons tout d'abord par calculer les enthalpies (incl. Gibbs) et entropies des réactions du réacteur 
primaire (vaporeformage). Nous calculons ensuite les constantes d'équilibres pour les deux réactions. 
Avec les hypothèses posées sur les réactions (toutes sont complètes exceptées celles s'effectuant dans le réacteur primaire),
et avec la masse désirée de \ce{NH3}, nous calculons les masses nécessaires de \ce{H2}, \ce{N2}. Puisque l'azote est fourni
par l'air, nous déterminons la masse de \ce{O2} fournie (ainsi que la masse de \ce{Ar}, mais l'argon est un gaz noble et 
n'intervient pas dans la réaction). 

Avec les hypothèses et la masse de \ce{O2}, nous déterminons la quantité de \ce{CH4} 
utilisé dans le réacteur secondaire et donc la production de \ce{H2} dans ce réacteur. Nous savons désormais 
déterminer la quantité de \ce{H2} à produire dans le réacteur primaire, et grâce à la constante d'équilibre déterminée 
plus tôt, nous pouvons déterminer la quantité de \ce{CH4} nécessaire pour les deux réactions. L'outil nous permet également
de calculer la quantité minimum de \ce{H2O} à fournir pour que toutes les équations se passent comme prévu. 
\\

Il est à noter que l'outil nous fourni des valeurs en tonnes par jour.

\section{Tubes}
Nous recherchons le nombre de tubes nécaissaire à l'acheminement du méthane et de l'eau dans le reformateur primaire.\\
\\
Soit $x$ le nombre de tubes, nous avons l'équation suivante :
\[
V'_{(m^3/s)} = A_{(m^2)}. c'_{(m/s)} .x
\]
Avec $V'$ le débit; $A$ la section d'un tuyau et $c'$ la vitesse superficielle à l'entrée du réacteur.\\
\\
Le volume peut être calculé grace à la loi des gaz parfaits;
\[
V=\frac{nRT}{P}
\]

Si l'on concidère notre calcul en un espace de temps d'1 seconde. La simplification des deux formules ci-dessus nous amène à
\[
x=\frac{nRT}{AcP}
\]
Ou la seule inconnue est les nombre de môles de produit rentrant, c'est a dire de $CH_{4}$ et de $H_{2}O$. Ce que nous avons pu déterminer grace au programme MATLAB sachant que 1500 T d'ammoniac était demandée à la sortie.\\
\\
$n_{CH_{4}}= 7,56_{kg} / 0,016_{kg/mole} = 472,18 moles$\\
$n_{H_{2}O}= 18,94_{kg/s}/ 0,016_{kg/mole} = 1052,22 moles$\\

Grâce aux autres valeurs données dans l'énoncé nous pouvons calculer le nombre de tubes nécessaire 


\[
x=\frac{(n_{CH_{4}}+n_{H_{2}O})8,314.1080}{\pi(0,05^2).2.(31.10^5)}=281,1
\]

Il faut donc 282 tuyaux pour assurer l'approvionnnement des composés dans le reformateur primaire.

\section{Bilan énergétique de la partie réformateur primaire-four}

Afin de connaître les quantités de \ce{CH4} et de \ce{O2} à introduire dans le four afin de garder le réformateur primaire à température constante nous devons étudier les réactions ayant lieu dans ce réformateur. Les calculs exposés dans cette section sont fait pour une température dans le réformateur de 1080K et pour une production journalière d'ammoniac de 1500T. Bien sur ces résultats sont facilement transposables à d'autres données grace au code MATLAB. \\
Tout d'abord nous allons calculer les enthalpies des différentes réactions qui nous intéressent: \\
\begin{itemize}
\item{\ce{CH4 + H2O <=> CO + 3H2} \\
$\Delta H_{reac1} (1080)=\int_{298.15}^{1080} Cp$ $dT = 226,94 kJ/mol$ (endothermique)}
\item{\ce{CO + H2O <=> CO2 + H2} \\
En appliquant la même formule on obtient: $\Delta H_{reac2} (1080) = -35,55 kJ/mol$ (exothermique)}
\item{\ce{CH4 + 2O2 -> CO2 + 2H2O} \\
En appliquant la même formule on obtient: $\Delta H_{reac3} (1080) = -803,56 kJ/mol$ (exothermique)}
\end{itemize}
\\
Après quelques calculs, on trouve que 478,17 moles de \ce{CH4} sont injectées par seconde dans le réformateur primaire, mais que seules 42.5\% de ces moles y "réagissent vraiment" (le reste va au secondaire).
Pour la 2ème réaction du réformateur primaire, seules 13,5\% des moles "réagissent vraiment". \\
De là, on peut calculer le bilan énergétique dans le réformateur primaire: \\
$\Delta U = (478,17 * 0,425) * (\Delta H_{reac1} (1080) + \Delta H_{reac2} (1080) * 0,135) = 45143,94 kJ/s$ \\
On sait que $\Delta H_{reac3} (1080) = -803,56 kJ/mol$, on peut en déduire le nombre de moles/s de \ce{CH4} dont on a besoin pour compenser le $\Delta U$ du réformateur primaire. \\
On obtient: n = 56,18 moles/s.
On sait que le rendement énergétique du four est de 75\%.
On aura donc besoin de 74,9 moles/s de \ce{CH4}, soit 1,2kg/s pour approvisionner le four. 

\end{document}
