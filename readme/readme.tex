\documentclass[a4paper,oneside,11pt]{article}

\usepackage{../custom}

\title{\textsc{readme} de l'outil de gestion}
\author{Groupe 1225}
\date{\today}

\newcommand{\fun}[1]{\texttt{#1}}

\begin{document}

\maketitle

\section{Utilisation}

Pour démarrer l’outil de gestion, il suffit de lancer la fonction \fun{startOutil} 
dans la ligne de commande \textsc{matlab}. 
Ceci va lancer l’\emph{interface graphique} 
et il ne reste plus qu’à introduire la quantité d’ammoniac (en tonnes par jour),
la température (en kelvin) et la pression (en bar) du reformeur primaire dans les cases appropriées.

\section{Fonctionnement}

Pour comprendre le fonctionnement de l'ensemble de l'outil,
on utilisera la commande \texttt{help X},
où \texttt{X} est une des fonctions suivantes.

\begin{itemize}
	\item \fun{analyseParametrique}
	\item \fun{efficiencePlot}
	\item \fun{environnement}
	\item \fun{equilibriumSimulation}
	\item \fun{getCoefficients} 
	\item \fun{getDeltaH\_and\_S}
	\item \fun{getEqConstantsRef} 
	\item \fun{getHovenMasses}
	\item \fun{getMassesDetails}
	\item \fun{getMolarMasses}
	\item \fun{getMolesDetails}
	\item \fun{getTubesNumber}
	\item \fun{main}
	\item \fun{printHovenDetails}
	\item \fun{printMassesDetails}	
	\item \fun{refroidissement}
	\item \fun{solveG}
	\item \fun{startOutil}
\end{itemize}

\paragraph{Remarque} Il se peut que pour certaines combinaisons de valeurs de $m_\ce{NH3}$,
et de $T$ une erreur de type ``\textit{Could not extract individual solutions. 
Returning a MuPAD set object.}'' survienne. 
Pour palier à ce problème il suffit de changer la valeur 
de $m_\ce{NH3}$ ou de $T$ d'une unité.

\end{document}
