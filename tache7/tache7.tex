\documentclass[a4paper, oneside, 12pt]{article}

\usepackage{../custom}

\title{Rapport des activités de terrain}
\author{Groupe 1225}
\date{12 Novembre 2014}

\begin{document}

\maketitle

\section{Visite de la station de biométhanisation de l’AIVE à Tenneville}

L’usine de biométhanisation permet de créer du CH4 à partir de déchets organiques,
principalement ménagers, mais aussi provenant de parcs à conteneurs, etc.

Premièrement, les déchets sont acheminés par camions au centre de Tenneville.
Avant que les déchets ne soient stockés dans un entrepôt, les camions passent à travers un portique afin 
d'y détecter la présence de déchets radioactifs. Ils sont ensuite pesés
puis déchargés dans le premier entrepôt.

Les déchets passent à travers un broyeur et sont acheminés par tapis roulant 
où des aimants permettent d’y retirer les déchets ferreux indésirables.
À cette étape, les déchets sont prêts à entrer dans le digesteur. 
Ce digesteur ressemble à un grand silo d’une capacité de $3000 \si{\meter\cubed}$.
Le principe est assez simple, des bactéries anaérobies vont commencer à digérer 
ces matières organiques en l’absence d’oxygène, ce qui crée du méthane. 
Le méthane sert à la production d’électricité et de chaleur (principe de cogénération).
L’usine est donc autosuffisante en termes d'électricité et de chauffage.
Le surplus  d’électricité est redistribué sur le réseau, et l’excédent de chaleur 
est utilisé pour le séchage de boue ainsi que dans d’autres usines de recyclage. 

Une fois la matière digérée, elle est compostée complètement jusqu’à l’obtention 
de terreau qui est revendu aux agriculteurs.
En cas de panne de moteur - ou tout autre problème dans le digesteur -
il y a une torchère qui permet du brûler tout surplus de méthane
et donc d'éviter tout excès dans le digesteur. Cela évite aussi de devoir 
libérer du méthane dans l’atmosphère.

Quelques chiffres :
\begin{itemize}
	\item Capacité de traitement : $39 000$ tonnes par an
	\item Production en électricité : équivalent à la consommation de $1500$ ménages
	\item Production de gaz : $55\%$ de \ce{CH4}, $44\%$ de \ce{CO2} + autres gaz
\end{itemize}

\section{Visite du centre Total Research Technology Feluy}

\subsection{Catalyse}
\label{subsec:catalyse}

L'utilisation d'un catalyseur est essentielle pour de nombreuses réactions industrielles 
effectuées aujourd'hui. De nombreuses réactions ne seraient même pas possibles sans 
l'utilisation d'un catalyseur.
 
Un catalyseur réduit l'énergie d'activation d'une réaction en affaiblissant les liens
électroniques des molécules devant réagir, ou en affaiblissant les liens entre les 
molécules des réactifs. Certains catalyseurs sont en pratique ``consommés'' durant 
la réaction (ex: emprisonnés dans les molécules) et d'autres ne sont simplement pas 
affectés par les réactions. 

Dans de nombreuses réactions, les catalyseurs ont également d'autres rôles. 
Suivant le catalyseur utilisé, certaines réactions seront favorisées au détriment 
d'autres (il faut donc trouver un catalyseur favorisant la réaction voulue). 
Le catalyseur peut également déterminer la structure des molécules obtenues lors 
d'une cristallisation. Suivant le catalyseur utilisé lors de la synthèse de polyéthylène,
on obtiendra une poudre fine ou de plus gros grains,
deux structures ayant des applications différentes.
 
Trouver le bon catalyseur est donc essentiel dans la chimie moderne.
 
\subsection{Unités Pilotes}
 
Le développement de nouveaux procédés ou catalyseurs commence tout d'abord en laboratoire,
ou de microréacteurs permettent de tester la viabilité des nouveaux développements.
Si un catalyseur ou un procédé est considéré comme intéressant,
il va ensuite être testé dans une unité pilote. 

Différentes réactions demandent différents réacteurs,
et le réacteur idéal pour une nouvelle réaction est déterminé en laboratoire.
 
Les unités pilotes sont des réacteurs industriels réduits utilisés pour tester de
nouvelles réactions ou de nouveaux procédés. Les unités pilotes sont beaucoup plus
modulables que les unités industrielles. Celles-ci vont permettre de détecter
d'éventuels problèmes qui sont passés inaperçus lors des tests en laboratoires,
ainsi que de déterminer les conditions idéales pour l'utilisation des nouvelles réactions.
Si un nouveau produit (nouvelle structure ou autre) est considéré comme intéressant,
les unités pilotes vont permettre de produire une quantité limitée de ce nouveau
produit afin de fournir des échantillons à des partenaires commerciaux.
Elles évitent ainsi de devoir reconfigurer des plants de grande taille.

\section{Laboratoire d’électrolyse}

L'électrolyse de l'eau est un procédé qui permet de décomposer 
celle-ci en dioxygène et en dihydroègne, tous deux à l'état gazeux.

\begin{equation*}
	\ce{H2O_{(g)} <=> \frac{1}{2} \, O2_{(g)} + H2_{(g)}} 
\end{equation*}

Afin de produire du dihydrogène gazeux en quantités décentes, l'électrolyse de l'eau 
demande énormément d'énergie électrique (sous forme de courant). De plus, le rendement
de la réaction d'électrolyse de l'eau ne dépasse en général jamais les $50\%$.

Sur base des expérimentations effectuées en laboratoire, nous observons que pour un même volume de dihydrogène produit, 
plus l’intensité du courant est élevée, plus le temps nécessaire à la production
du dihydrogène est faible. En fait, le courant et le temps sont inversément proportionnels, 
et nous pouvons écrire la relation suivante :

\begin{equation*}
	It = \text{constante}
\end{equation*}

où I est l'intensité du courant électrique, et t, le temps écoulé. 
Cela montre bien que lorsqu'on augmente le courant, le temps diminue dans les mêmes proportions.

Calculons maintenant la puissance nécessaire pour produire tout l'hydrogène 
dont nous avons besoin pour un débit d'ammonicac de 1500 t/j.

En prenant une température de $1000 \si{\kelvin}$ pour le reformeur primaire, 
nous savons grâce à notre outil de calcul que pour produire 1500 t/j d'ammoniac, il faut $266,32 \si{\tonne}$ de dihydrogène 
par jour. Cela nous donne un débit massique de $184,944 \, \si{\kilo\gram/\minute}$.

Nous devons maintenant fixer trois paramètres pour le déroulement de la réaction d'électrolyse : le pH, la température et le courant.
La réaction d'électrolyse nécessite une puissance moins importante lorsque le pH tend vers 0. Mais un pH nul demande une grande quantité d'acide, nous allons donc nous contenter ici d'un pH = 1.
Par souci d'économie d'énergie, nous allons travailler à température ambiante, soit $20\si{\celsius}$.
Le dernier paramètre est le courant, que nous allons faire varier afin d'obtenir la quantité voulue d'hydrogène.
Par exemple, en appliquant un courant de $0,5 \si{\ampere}$, nous savons, grâce aux expériences faites en laboratoire, 
que nous produisons $4 \, \si{\milli\liter}$ de dihydrogène par minute. 
Considérons maintenant que le dihydrogène se comporte comme un gaz parfait, nous pouvons appliquer :

\begin{equation*}
	pV = mR^{*}T
\end{equation*}

et donc calculer la masse de dihydrogène, sachant que la réaction se déroule 
sous une température de $293,15 \si{\kelvin}$ et une pression de $1 \si{\bar}$. 
Nous obtenons ainsi qu'en faisant circuler un courant de $0.5 \si{\ampere}$, 
nous produisons $3,2824\e{-7} \, \si{\kilo\gram/\minute}$ de dihydrogène.

Pour atteindre un débit massique de $184,944 \, \si{\kilo\gram/\minute}$, 
il faut donc un courant électrique de $2,817\e{8} \si{\ampere}$. 
Nous savons grâce aux résultats du laboratoire que la tension électrique dépend quant à elle du pH. 
Pour un pH = 1, la tension appliquée par la source sera de 10 V.
Nous pouvons alors calculer la puissance électrique grâce à la relation suivante : 

\begin{equation*}
	P = UI
\end{equation*}

La puissance est donc égale à $2,817 \, \si{\giga\watt}$.

Cette puissance requise est énorme. L'électrolyse de l'eau demande 
beaucoup trop de puissance pour produire ce dont nous avons besoin. 
C'est pour cela que l'électrolyse est encore très peu utilisée à l'échelle industrielle. 
Il est donc préférable dans notre cas de se contenter du vaporeformage, 
malgré le fait qu'il émette du dioxyde de carbone.

\section{Visite du plant de Yara à Tertre}

Le plant de Yara fabrique des engrais et produit lui-même son propre ammoniac nécessaire
pour créer ces engrais. Avoir une production locale d'ammoniac permet non seulement 
d'éviter des coûts importants mais aussi d'éviter le transport de celui-ci car il 
peut s'avérer être toxique.

Tout d'abord, le réacteur où est synthétisé l'ammoniac est à une pression de $130 \si{\bar}$.
Comme décrit dans la section \ref{subsec:catalyse}, l'utilisation de catalyseurs dans la
production du méthane est fondamentale. Un catalyseur est utilisé lorsque la 
barrière énergétique pour qu'une réaction se produise est trop haute. Il va alors 
diminuer l'énergie d'activation sans modifier le chemin réactionnel.

Le diagramme de la figure \ref{fig:synthese} présente le fonctionnement simplifié 
des réacteurs de synthèse d'ammoniac. Il est important de préciser que le \emph{input}
est tout d'abord composé d'hydrogène et d'azote mais aussi de résidus de méthane, 
d'argon et d'hélium. Ceux-ci sont éliminés via différentes techniques 
de séparation (ex: par des membranes semi-perméables). 

\begin{figure}[h!]
	\begin{center}
		\begin{tikzpicture}
[node distance = 7em]

\tikzstyle{block} = [rectangle, draw, fill=blue!20, 
    text width=10em, text centered, rounded corners, minimum height=3em]
\tikzstyle{block2} = [rectangle, draw, fill=red!60, 
    text width=6em, text centered, minimum height=2em]
\tikzstyle{line} = [draw, -latex']
\tikzstyle{cloud} = [draw, ellipse,fill=red!20, node distance=2cm,
    minimum height=1em]
    
\node [block2] (purge) {Purge};
\node [block, below = 4em of purge] (compresseur) {Compresseur/Synthèse};
\node [cloud, right = 10em of purge] (input) {Input};
\node [block, right = 6em of compresseur] (cooler) {Refroidisseur};
\node [block, below = 4em of cooler] (separation) {Separation};
\node [cloud, left = 8em of separation] (output) {Output};

\path [line] (input) -- node[anchor = south] {\ce{N2} \ce{H2} \ce{CH4} \ce{Ar} \ce{He}}
	(purge);
\path [line] (purge) -- node[anchor = west] {\ce{N2} \ce{H2}}
	(compresseur);
\path [line,dashed] (compresseur) -- node[anchor = south] {\ce{NH3_{(g)}} \ce{NH3_{(l)}}} 
	(cooler);
\path [line,dashed] (cooler) -- node[anchor = west] {\ce{NH3_{(g)}} \ce{NH3_{(l)}}} 
	(separation);
\path [line,dashed] (separation) -- node[anchor = east] {\ce{NH3_{(g)}}}
	(compresseur);
\path [line] (separation) -- node[anchor = north] {\ce{NH3_{(l)}}} 
	(output);
\end{tikzpicture}


	\end{center}
	\caption{Fonctionnement simplifié de la synthèse d'ammoniac.}
	\label{fig:synthese}
\end{figure}

Une autre information importante concerne la quantité d'énergie nécessaire pour produire
une tonne d'ammoniac. Théoriquement on devrait avoir besoin de $20 \si{\giga\joule/\tonne}$
pour l'ensemble du processus. On observe cependant qu'il nous faut 
pratiquement $30 \si{\giga\joule/\tonne}$. On remarque donc qu'une partie 
importante de l'énergie est perdue à travers les différentes transformations. 
Ces pertes peuvent provenir par exemple d'une qualité non optimale (d'un point de vue 
de rendement) des réacteurs ou de tuyauteries peu isolantes thermiquement.

Le dernier point essentiel de la visite concerne la partie sur l'impact environnemental 
de la société. On compte environ $1000$ tonnes par jour de \ce{CO2} produit, une moitié,
dite impure, sera rejetée dans l'atmosphère tandis que l'autre moitié, dite pure, 
sera reliquéfiée pour ensuite être revendue (utile pour la production de bière par exemple).
On notera aussi le rejet de composés très polluants tels que le \ce{NO2} et le \ce{NOx} 
qui proviennent des fumées du reformage primaire à cause d'un manque d'efficacité
du brûleur.

\section{Atelier créatif (conduite de brainstorming)}

Comment faire preuve d'inventivité et d'originalité dans un projet tel que le nôtre ?
Tel était le thème de cet atelier où l'on nous a présenté le cheminement 
à suivre afin de mettre un maximum à profit la créativité de chacun.

Dans notre cas, la créativité est avant tout l'art de trouver des solutions 
originales et efficaces à un problème bien posé. 
La première partie du travail consiste donc à reformuler la problématique de façon 
à être capable de diverger et de trouver un angle nouveau sous lequel analyser le problème. 
Pour cela, il est conseillé de faire un schéma sur lequel on pourra retrouver différents
éléments comme par exemple les services attendus, la position géographique, 
ce qui se situe à proximité de l’usine,etc.

Cette représentation permet d'avoir une vue d'ensemble sur le travail qui 
devra être effectué et facilite donc l'organisation. Il est important de pousser tout
le monde à sortir de sa ``zone de confort'' et de confronter les idées de chacun.
Vient ensuite le retour à la réalité: il faut analyser les idées, choisir celles qui 
sont réalisables pour que le projet se précise enfin. De nombreux exercices ont été 
proposés afin de nous mettre en situation. 

Nous pouvons à présent faire bénéficier le reste du groupe de notre expérience et tenter
de suivre cette approche. Il est nécessaire de faire un résumé de tout cela pour essayer 
de convaincre et de prouver la viabilité de l'ébauche. Pour cela, on fait la liste 
de quatre choses: les besoins du client, la promesse, les raisons d'y croire, 
et un slogan pour accrocher. On continue l'argumentation avec des analogies,
des choses factuelles (brevets,...) ou inspirantes, etc. Le projet pourra alors
enfin être mis en place!

Une présentation sur le développement durable a aussi été faite afin de nous sensibiliser
et nous inviter à essayer d'y participer. 
En effet, notre système industriel actuel - où la recherche du profit occupe une position
centrale - est voué à l'échec, car il mène à des crises dans de nombreux domaines.
C'est pourquoi la société doit se remettre en question et évoluer pour atteindre
l'idéal d'un système organique, où l'humanité travaillerait ensemble pour un but commun. 

Dans notre cas, cela se résume à se préoccuper de l'écologie mais pas uniquement:
il serait intéressant de collaborer avec d'autres sociétés au niveau de l'importation
des ressources et de l'exportation nos déchets. Ces derniers pourraient être utiles
à d'autres et nous pourrions ainsi trouver des ententes profitables pour tous,
s'approchant d'un système cyclique.


\appendix
\documentclass[12pt,oneside]{article}

\NeedsTeXFormat{LaTeX2e}
\ProvidesPackage{custom}[2014/05/11 Custom Package]

\usepackage[utf8]{inputenc}
\usepackage[T1]{fontenc}
\usepackage[francais]{babel}

\usepackage[version=3]{mhchem}
\usepackage{chemfig}

\usepackage{amsmath}
\usepackage{amsthm}
\usepackage{amsfonts}

\usepackage{graphicx} 
\usepackage[top=3cm, bottom=3cm, left=3cm , right=3cm]{geometry}
%\usepackage{setspace} %doublespace, onehalfspace
\usepackage{siunitx}

\usepackage{tikz}
\usetikzlibrary{positioning}
\usetikzlibrary{shapes,arrows}

\usepackage{tabularx}
\usepackage{url} 
\usepackage{tocloft} %spacing in list of figures
\usepackage{listings} %input code
%\usepackage{multibbl} %multiplebibliography
\usepackage{hyperref}
\usepackage[babel=true]{csquotes}
\usepackage{listings}
\usepackage{color}

\usepackage{epstopdf}

\usepackage{caption}
\usepackage{subcaption}
\usepackage{float}

\newcommand{\dif}[1]{\mathrm{d}#1}
\newcommand{\e}[1]{\cdot 10^{#1}}

\endinput


\title{Annexe 3 - Préparation du laboratoire d'électrolyse}
\author{Groupe 1225}
\date{Lundi 3 Novembre}

\begin{document}

\maketitle

Les valeurs exactes demandées ainsi que le graphique de l'évolution de l'équilibre
dans la solution en acide sulfurique en fonction du pH sont obtenus en
exécutant la fonction \texttt{electrolyse.m}.

\section{Solution à pH acide}

Nous devons déterminer le volume (en \si{\milli\liter}) de \ce{H2SO4} (5M)
nécessaire pour atteindre un certain pH.
Il s'agit donc de prendre en compte le fait qu'une fois que tout 
le \ce{H2SO4} sera transformé en \ce{HSO4-}, celui-ci à son tour contribuera
à l'augmentation du pH étant donné que c'est un acide faible.

Les deux réactions sont les suivantes

\begin{align*}
	\ce{H2SO4 + H2O <-> HSO4- + H3O+} \\
	\ce{HSO4- + H2O <-> SO4^{2-} + H3O+}
\end{align*}

On connait le pH de la solution, $K_{a1}$\footnote{On observe que malgré le fait 
que $K_{a1}$ soit très grand, la réaction n'est pas totalement complète et il est
donc important de calculer l'équilibre pour la première fonction de \ce{H2SO4}.}
et $K_{a2}$ qui sont les constantes d'équilibre associées respectivement 
à la première et la seconde réaction.

On obtient donc un système de trois équations où les trois inconnues
sont $x$ le nombre de moles initial de \ce{H2SO4}, $\xi_1$ et $\xi_2$
les degrés d'avancement des réactions.

\begin{align*}
	K_{a1} &= \frac{\xi_1^2}{x - \xi_1} \\
	K_{a2} &= \frac{\xi_2 \, (\xi_1 + \xi_2)}{\xi_1 - \xi_2} \\
	\text{pH} &= - \log_{10}{(\xi_1 + \xi_2)} 
\end{align*}

Ce système est résolu dans \texttt{electrolyse.m}.

\section{Solution à pH basique}

Nous devons à nouveau déterminer le volume (en \si{\milli\liter}) 
de \ce{NaOH} (5M) nécessaire pour atteindre un certain pH.
Cette partie est beaucoup plus simple étant donné que l'hydroxide de sodium
est une base forte et n'a qu'une seule fonction basique.

Par la relation suivante

\[
	[\ce{OH-}] \cdot [\ce{H3O+}] = 10^{-14} 
\]
On trouve très facilement que 

\[
	V_{\ce{NaOH}} = \frac{10^{-14}}{10^{- \text{pH}}} \cdot 200 
	\quad \text{(en \si{\milli\liter})}
\]



\end{document}


\end{document}

