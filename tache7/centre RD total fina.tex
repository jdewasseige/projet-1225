\documentclass[12pt,oneside]{article}

\NeedsTeXFormat{LaTeX2e}
\ProvidesPackage{custom}[2014/05/11 Custom Package]

\usepackage[utf8]{inputenc}
\usepackage[T1]{fontenc}
\usepackage[francais]{babel}

\usepackage{mhchem}
\usepackage{chemfig}
\usepackage{amsmath}
\usepackage{amsthm}
\usepackage{amsfonts}

\usepackage{graphicx} 
\usepackage[top=3cm, bottom=3cm, left=3cm , right=3cm]{geometry}
%\usepackage{setspace} %doublespace, onehalfspace
\usepackage{siunitx}
\usepackage{tikz}
\usepackage{tabularx}
\usepackage{url} 
\usepackage{tocloft} %spacing in list of figures
\usepackage{listings} %input code
%\usepackage{multibbl} %multiplebibliography
\usepackage{hyperref}
\usepackage[babel=true]{csquotes}
\usepackage{listings}
\usepackage{color}

\newcommand{\mi}{\mathrm{i}} %symbole complexe i
\newcommand{\mj}{\mathrm{j}} %symbole complexe j
\newcommand{\dif}[1]{\mathrm{d}#1}
\newcommand{\e}[1]{\cdot 10^{#1}}

\endinput


\title{Annexe 4 - Visite du Centre Recherche et développement Total Fina}
\author{Groupe 1225}
\date{Mardi 4 Novembre}

\begin{document}

\maketitle
 
 \section{Catalyse}
 
 L'utilisation d'un catalyseur est essentiel pour de nombreuses réactions industrielles effectuées aujourd'hui. De nombreuses
 réactions ne seraient même pas possibles sans l'utilisation d'un catalyseur.
 
 Un catalyseur réduit l'énergie d'activation d'une réaction en affaiblissant les liens électroniques des molécules devant 
 réagir, ou en affaiblissant les liens entre les molécules des réactifs. Certains catalyseurs sont en pratique ``consommés'' 
 durant la réaction (ex: emprisonnés dans les molécules) et d'autres ne sont simplement pas affectés par les réactions. 
 Dans de nombreuses réactions, les catalyseurs on également d'autres roles. Suivant le catalyseur utilisé, certaines 
 réactions seront favorisées aux détriments de d'autres (il faut donc trouver un catalyseur favorisant la réaction voulue)
 . Le catalyseur peut également déterminer la structure des molécules obtenues lors d'une crystallisation. Suivant le catalyseur
 utilisé lors de la synthèse de polyéthylène, on obtiendra une poudre fine ou de plus gros grains, deux structures ayant des
 applications différentes.
 
 Trouver le bon catalyseur est donc essentiel dans la chimie moderne.
 
 \section{Unités Pilotes}
 
 Le développement de nouveaux procédés ou catalyseurs commence tout d'abord en laboratoire, ou de micro réacteurs permettent de tester la viabilité 
 des nouveaux développements. Si un catalyseur ou un procédé est considéré comme intéressant, il va ensuite être testé dans une unité pilote. 
 Différentes réactions demandent différents réacteurs, et le réacteur idéal pour une nouvelle réaction est déterminé en laboratoire.
 
 Les unités pilotes sont des réacteurs industriels réduits utilisés pour tester de nouvelles réactions ou procédés. Elles sont beaucoup plus modulables
 que les unités industrielles. Celles-ci vont permettre de détecter d'éventuels problèmes qui sont passés inaperçus lors des tests en laboratoires, 
 ainsi que de déterminer les conditions idéales pour l'utilisation des nouvelles réactions. Si un nouveau produit (nouvelles structure ou autre) est
 considérée comme intéressante, les unités pilotes vont également permettre de produire une quantité limitée de ce nouveau produit afin de fournir
 des échantillons à des partenaires commerciaux. Elles évitent ainsi de devoir reconfigurer des plants de grande taille.
 
\end{document}

