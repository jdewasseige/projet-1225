\documentclass[12pt,oneside]{article}

\NeedsTeXFormat{LaTeX2e}
\ProvidesPackage{custom}[2014/05/11 Custom Package]

\usepackage[utf8]{inputenc}
\usepackage[T1]{fontenc}
\usepackage[francais]{babel}

\usepackage{mhchem}
\usepackage{chemfig}
\usepackage{amsmath}
\usepackage{amsthm}
\usepackage{amsfonts}

\usepackage{graphicx} 
\usepackage[top=3cm, bottom=3cm, left=3cm , right=3cm]{geometry}
%\usepackage{setspace} %doublespace, onehalfspace
\usepackage{siunitx}
\usepackage{tikz}
\usepackage{tabularx}
\usepackage{url} 
\usepackage{tocloft} %spacing in list of figures
\usepackage{listings} %input code
%\usepackage{multibbl} %multiplebibliography
\usepackage{hyperref}
\usepackage[babel=true]{csquotes}
\usepackage{listings}
\usepackage{color}

\newcommand{\mi}{\mathrm{i}} %symbole complexe i
\newcommand{\mj}{\mathrm{j}} %symbole complexe j
\newcommand{\dif}[1]{\mathrm{d}#1}
\newcommand{\e}[1]{\cdot 10^{#1}}

\endinput


\title{Annexe 3 - Préparation du laboratoire d'électrolyse}
\author{Groupe 1225}
\date{Lundi 3 Novembre}

\begin{document}

\maketitle

Les valeurs exactes demandées ainsi que le graphique de l'évolution de l'équilibre
dans la solution en acide sulfurique en fonction du pH sont obtenus en
exécutant la fonction \texttt{electrolyse.m}.

\section{Solution à pH acide}

Nous devons déterminer le volume (en \si{\milli\liter}) de \ce{H2SO4} (5M)
nécessaire pour atteindre un certain pH.
Il s'agit donc de prendre en compte le fait qu'une fois que tout 
le \ce{H2SO4} sera transformé en \ce{HSO4-}, celui-ci à son tour contribuera
à l'augmentation du pH étant donné que c'est un acide faible.

Les deux réactions sont les suivantes

\begin{align*}
	\ce{H2SO4 + H2O <-> HSO4- + H3O+} \\
	\ce{HSO4- + H2O <-> SO4^{2-} + H3O+}
\end{align*}

On connait le pH de la solution, $K_{a1}$\footnote{On observe que malgré le fait 
que $K_{a1}$ soit très grand, la réaction n'est pas totalement complète et il est
donc important de calculer l'équilibre pour la première fonction de \ce{H2SO4}.}
et $K_{a2}$ qui sont les constantes d'équilibre associées respectivement 
à la première et la seconde réaction.

On obtient donc un système de trois équations où les trois inconnues
sont $x$ le nombre de moles initial de \ce{H2SO4}, $\xi_1$ et $\xi_2$
les degrés d'avancement des réactions.

\begin{align*}
	K_{a1} &= \frac{\xi_1^2}{x - \xi_1} \\
	K_{a2} &= \frac{\xi_2 \, (\xi_1 + \xi_2)}{\xi_1 - \xi_2} \\
	\text{pH} &= - \log_{10}{(\xi_1 + \xi_2)} 
\end{align*}

Ce système est résolu dans \texttt{electrolyse.m}.

\section{Solution à pH basique}

Nous devons à nouveau déterminer le volume (en \si{\milli\liter}) 
de \ce{NaOH} (5M) nécessaire pour atteindre un certain pH.
Cette partie est beaucoup plus simple étant donné que l'hydroxide de sodium
est une base forte et n'a qu'une seule fonction basique.

Par la relation suivante

\[
	[\ce{OH-}] \cdot [\ce{H3O+}] = 10^{-14} 
\]
On trouve très facilement que 

\[
	V_{\ce{NaOH}} = \frac{10^{-14}}{10^{- \text{pH}}} \cdot 200 
	\quad \text{(en \si{\milli\liter})}
\]



\end{document}
