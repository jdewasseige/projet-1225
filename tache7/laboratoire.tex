\section{Préparation du laboratoire d'électrolyse}

Les valeurs exactes demandées ainsi que le graphique de l'évolution de l'équilibre
dans la solution en acide sulfurique en fonction du pH sont obtenus en
exécutant la fonction \texttt{electrolyse.m}.

\subsection{Solution à pH acide}

Nous devons déterminer le volume (en \si{\milli\liter}) de \ce{H2SO4} (5M)
nécessaire pour atteindre un certain pH.
Il s'agit donc de prendre en compte le fait qu'une fois que tout 
le \ce{H2SO4} sera transformé en \ce{HSO4-}, celui-ci à son tour contribuera
à l'augmentation du pH étant donné que c'est un acide faible.

Les deux réactions sont les suivantes

\begin{align}
	\ce{H2SO4 + H2O -> HSO4- + H3O+} \nonumber \\
	\ce{HSO4- + H2O <=> SO4^{2-} + H3O+} \label{eq:chemHSO4}
\end{align}

On connait le pH de la solution, $K_{a1}$ et $K_{a2}$ qui sont les constantes 
d'équilibre associées respectivement à la première et la seconde réaction.

La première réaction est complète car \ce{H2SO4} est un acide plus fort que \ce{H3O+}.
On obtient donc un système de deux équations où les deux inconnues
sont $x$ le nombre de moles initial de \ce{H2SO4} et $\xi$
le degré d'avancement de la réaction \ref{eq:chemHSO4}.

\begin{align*}
	K_{a2} &= \frac{\xi \, (x + \xi)}{x - \xi} \\
	\text{pH} &= - \log_{10}{(x + \xi)} 
\end{align*}

Ce système est résolu dans \texttt{electrolyse.m}.

\subsection{Solution à pH basique}

Nous devons à nouveau déterminer le volume (en \si{\milli\liter}) 
de \ce{NaOH} (5M) nécessaire pour atteindre un certain pH.
Cette partie est beaucoup plus simple étant donné que l'hydroxide de sodium
est une base forte et n'a qu'une seule fonction basique.

Par la relation suivante

\[
	[\ce{OH-}] \cdot [\ce{H3O+}] = 10^{-14} 
\]
On trouve très facilement que 

\[
	V_{\ce{NaOH}} = \frac{10^{-14}}{10^{- \text{pH}}} \cdot 200 
	\quad \text{(en \si{\milli\liter})}
\]



